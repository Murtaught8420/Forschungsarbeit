

\chapter{Konstruktiver Entwicklungsprozess des Prüfstandes}\label{cha:3}

Für die vorliegende Arbeit soll soweit möglich der in der Fachliteratur wie bspw. der Konstruktionslehre nach Pahl und Beitz~\cite{Pahl_Beitz_Konstruktionslehre} oder der VDI~2222~\cite{VDI_2222-1} schematisierte kontruktive Entwicklungsprozess Anwendnung finden. Die Darstellung des Entwicklungsprozesses erfolgt weiterhin in Anlehung an~\cite{Projektarbeit}.
\par
\vspace{\linespace}
Der Synthese einer konstruktiven Lösung geht die Analyse der Aufgabenstellung hinsichtlich der gestellten Anforderungen voraus, deren Ergebnis eine Anforderungsliste ist. Dadurch wird sichergestellt, dass der Entwicklungsprozess alle Forderungen erfüllt. Das in der anschließenden Konzeptphase entstandene Konzept wird dann im Verlauf der Entwurfsphase in eine quantitative Gestaltung überführt. Im Rahmen der letzten Entwicklungsphase, der Ausarbeitung, werden die verbliebenen Feinheiten zur Erfüllung der Anforderungen eingearbeitet. Die fertig ausgearbeitete Lösung schließt den Entwicklungsprozess ab.
\par
\vspace{\linespace}
Eine klare Trennung der beschriebenen Phasen kann in der Praxis jedoch selten erfolgen. Hinzu kommt das Durchlaufen mehrerer Iterationsschleifen, bei denen die gestellten Anforderungen stets durch neu erhaltene Informationen und die vorangeschrittene Entwicklung weiter präzisiert und soweit wie möglich mit konkreten Werten belegt werden. Im Folgenden soll das Ergebnis des konstruktiven Entwicklungsprozesses kurz dargestellt und erläutert werden.   


\section{Präzisierung der Aufgabenstellung}\label{cha:3_Praezisierung_Aufgabenstellung}




Im Verlauf der Aufgabenanalyse erfolgt die Erstellung einer Anforderungsliste, welche die Forderungen an die Konstruktion möglichst umfangreich zusammenstellt. Die Anforderungen setzen sich hauptsächlich aus den Rahmenbedingungen und Wünschen hinsichtlich verschiedener Hauptmerkmale und zur Funktionserfüllung notwendigen Forderungen zusammen.
\par
\vspace{\linespace}
Angelehnt an die Kategorisierung, wie sie in der Literatur beschrieben wird, wurde die Unterteilung in \acp{F} und \acp{W} vorgenommen. Dadurch konnten Anforderungen identifiziert werden, die durch alle Konzepte unbedingt zu erfüllen sind und solche, die nach Möglichkeit Berücksichtigung finden sollten. Der Erfüllungsgrad aller Anforderungen bildete die Bewertungsgrundlage der Konzeptideen.
\par
\vspace{\linespace}
Die Präzisierung der Detailanforderungen erfolgte teilweise und insofern möglich in Iterationsschleifen während der Konzept- und Entwurfsphase, wobei hier nur das Ergebnis dargestellt werden soll. Die Anforderungsliste im \Anhang\ref{A:Anforderungsliste} enthält bei Forderungen, die im Verlauf des Entwicklungsprozesses konkretisiert wurden, zusätzlich und falls notwendig eine Bemerkung zur Hauptquelle für die Präzisierung.









\section{Konstruktion kommerzieller Absorberkammern}\label{cha:3_Konstruktion_kommerzieller_Absorberkammern}



Geschirmte Räume bzw. Absorberkammern werden im kommerziellen Sektor von vielen Herstellern benötigt, da der Test auf elektromagnetische Verträglichkeit von Betriebsmitteln u.a. Teil der CE"=Kennzeichnung~\cite{Richtlinie_2014/30/EU} ist. Gängig sind dabei große Räume, die oft von unabhängigen Testzentren betrieben werden und eine große Bandbreite von Geräten und Betriebsmitteln testen können. Die Testräume bestehen dabei im Allgemeinen aus einzelnen Modulen oder die Schirmung wird direkt in die Wand eines Gebäudes integriert~\cite{EM_Schirmung}. Im Folgenden wird die Modulbauweise näher betrachtet.
\par
\vspace{\linespace}
Grundsätzlich sind zwei Bauweisen von Schirmkabinen üblich: Gekantete Stahlbleche, die miteinander verschraubt werden, und Verbundplatten aus metallischen Deckblechen mit einem Kern aus Pressspan, die durch Profile miteinander verbunden werden~\cite{EM_Schirmung, Design_of_shielded_enclosures}. In \Abb\ref{fig:3_Allgemeiner_Aufbau_Schirmkabinen} sind beide Varianten schematisch dargestellt. Alle im Rahmen der Recherche betrachteten Anbieter geschirmter Räume bzw. Messkabinen verwendeten eine dieser Varianten ohne nennenswerte Abwandlung.

\begin{figure}[H]
    \centering
    \begin{subfigure}[b]{0.4\textwidth}
        \includegraphics[page = 2, height=0.145\textheight,trim = 8.5cm 5.5cm 8.5cm 4.5cm, clip]{Abbildungen/Kapitel3/Schirmkabinen.pdf}
        \caption{Stahlblechplatten\label{subfig:3_Aufbau_Stahlblechkabine}}
    \end{subfigure}
    \hspace{1cm}
    \begin{subfigure}[b]{0.5\textwidth}
        \includegraphics[page = 3, height=0.135\textheight, trim = 11.5cm 6.5cm 7cm 6cm, clip]{Abbildungen/Kapitel3/Schirmkabinen.pdf}
        \caption{Sandwichmodule\label{subfig:3_Aufbau_Sandwichkabine}}
    \end{subfigure}
    %\includegraphics[width=.8\textwidth]{Abbildungen/Kapitel2/Feldverlauf.png}
    \caption[Schematischer Aufbau von Schirmkabinen]{Schematischer Aufbau von Schirmkabinen nach~\cite{EMC-Technik_Stahlblechplatten, EMC-Technik_Sandwichmodul, EM_Schirmung, Design_of_shielded_enclosures}}
    \label{fig:3_Allgemeiner_Aufbau_Schirmkabinen}
\end{figure}

Der Vorteil verschraubter Stahlblechwannen ist offensichtlich die hohe Haltbarkeit der Verschraubung. Allerdings ist die erreichbare Schirmdämpfung sehr sensitiv bezüglich der korrekten Lage der Hochfrequenz"=Dichtungen (HF-Dichtungen) zwischen den Blechplatten. Weiterhin neigen diese Module zum Vibrieren und je nach Größe der einzelnen Module ist zusätzlich eine Außenkonstruktion für die Selbsttragfähigkeit des Schirmkabine notwendig~\cite{EM_Schirmung}.
\par
\vspace{\linespace}
Ein System aus Verbund- oder Sandwich-Paneelen ist selbsttragend und benötigt aufgrund der überlappenden Profile zwischen den einzelnen Modulen keine HF-Dichtungen. Als Nachteil ist hier vor allem zu nennen, dass die unsachgemäße Verschraubung bei Verwendung selbstschneidender Schrauben zu Lecks führen kann, weil die Anpressung der Profile an die Modulwände stellenweise nicht mehr gegeben ist~\cite{EM_Schirmung}. 
\par
\vspace{\linespace}
Bei vergleichbarer Größe und Ausstattung sind Kabinen aus Sandwich-Modulen im Allgemeinen günstiger als solche aus Stahlblechplatten~\cite{EMC-Technik_Sandwichmodul, EMC-Technik_Stahlblechplatten}. Deshalb und aufgrund der Sensitivität gegenüber der korrekten Lage der HF-Dichtungen soll die Konstruktion im Rahmen dieser Arbeit in Anlehnung an die Sandwich-Modulbauweise erfolgen. 








\section{Konzepterstellung}\label{cha:3_Konzepterstellung}




\subsection{Funktionsstruktur und Konzeptbewertung}

Ein Konzept stellt in erster Linie einen prinzipiellen Aufbau dar und dient der Überprüfung, inwieweit die gestellten Anforderungen erfüllt wurden. Da durch das Konzept die grundlegende Gestalt und die wichtigsten Merkmale festgelegt werden, ist die Konzeptphase einer der wichtigsten Abschnitte im Entwicklungsprozess. Das Gesamtkonzept lässt sich nach~\cite{Pahl_Beitz_Konstruktionslehre} folgendermaßen unterteilen:

\begin{description}
    \item[Gestaltungskonzept] Festlegung der grundlegenden Geometrie, Zuordnung der einzelnen Komponenten und Betrachtung von Stoff-, Energie- und Signalflüssen
    \item[Wirkkonzept] Beschreibung der physikalischen Effekte und deren Verknüpfung zur Erfüllung der gestellten Anforderungen sowie der Gestalt der Wirkflächen
    \item[Wirkfläche] Funktionale Flächen, an denen das Wirkkonzept umgesetzt wird
\end{description}

Unter Berücksichtigung der Forderungen, welche die Funktion unmittelbar beeinflussen, wurden zunächst die wesentlichen Aufgaben der Konstruktion identifiziert. Die Gesamtaufgabe ist bereits eindeutig festgelegt: Innerhalb einer von äußeren Störeinflüssen abgeschirmten Messkabine sollen Probekörper im Koppelpfad zweier Hornantennen positioniert werden, wobei durch die geometrischen Abmessungen und die Verkleidung reflektiver Flächen im Inneren der Messkammer sichergestellt werden soll, dass sich die Proben im ungestörten Fernfeld (vgl. \Kapitel\ref{cha:2}) der Antennen befinden. Festzulegen bleibt im Weiteren der genaue Aufbau der Messkabine unter Berücksichtigung der zu erreichenden Schirmung, die Gestaltung von Öffnungen und Durchführungen, die Positionierung der Antennen und Probekörper und die Auswahl geeigneter Absorber, um Hohlraumresonanzen (vgl. \Abschnitt\ref{cha:2_subsub_Hohlraumresonanzen}), Reflektionen und indirekte Kopplung (vgl. \Abschnitt\ref{cha:2_sub_Reflektion}) zu vermeiden. Mithilfe der in \Abb\ref{fig:3_Funktionsstruktur} dargestellten Funktionsstruktur wurde die Gesamtaufgabe weiter abstrahiert und gegliedert. %Die Qualifikation des Teststandes anhand des Vergleiches der durchgeführten und mit früheren Messungen ist als Teil der Auswertung nicht in der Funktionsstruktur enthalten.

\begin{figure}[ht]
    \centering
    \includegraphics[page = 1, width=\textwidth, trim = 0.8cm 0.5cm 1.3cm 5.5cm, clip]{Abbildungen/Kapitel3/Funktionsstruktur.pdf}
    \caption{Funktionsstruktur der Messkabine}
    \label{fig:3_Funktionsstruktur}
\end{figure}


Auf dieser Grundlage lassen sich nun verschiedene Wirkkonzepte für die Teilstrukturen der Messkammer und deren Wirkflächen erstellen. Das Augenmerk des Designs lag dabei auf einer robusten und einfachen Geometrie der Wirkflächen bei gleichzeitiger Gewährleistung einer möglichst hohen Dichtheit gegenüber elektromagnetischer Strahlung. Letzteres kann nach der betrachteten Theorie in den \Abschnitten\ref{cha:2_sub_Daempfung_und_Absorption}, \ref{cha:2_sub_Reflektion} und \ref{cha:2_sub_Schirmung_ebener_Wellenfelder} durch die Herstellung eines leitfähigen Flächenkontaktes mit möglichst geringem Kontaktwiderstand zwischen allen Elementen der äußeren Schirmwand erreicht werden. Dies ist somit der physikalische Effekt, auf dem die Wirkkonzepte der Messkabine und der Durchführungen beruhen. 
\par
\vspace{\linespace}
Zur strukturierten und nachvollziehbaren Durchführung des Entscheidungsprozesses und der Auswahl der potenziell besten aus den vorgestellten Lösungsvarianten, kam ein Bewertungsschema zum Einsatz. Die Bewertungskriterien wurden nach dem Top-down-Vorgehen abgeleitet. Das beudeutet die Anforderungen des gesamten Versuchsstandes wurden für die einzelnen Teilsysteme detailliert~\cite{Pahl_Beitz_Konstruktionslehre}. Die gewählten Kriterien weisen die in~\cite{Pahl_Beitz_Konstruktionslehre} beschriebenen Voraussetzungen, wie unter anderem Freiheit von Dopplungen, Gegenläufigkeit und Widersprüchen sowie Gültigkeit für alle vorgestellten Varianten, auf.
\par
\vspace{\linespace}
Die Bewertung bestand aus der Anwendung eines gewichteten Punkteschemas, welches allen Kriterien Maßzahlen hinsichtlich ihrer Erfüllung durch die Konzepte zuordnete. Dies sorgte für ein vergleichsweise einfaches Bewertungsschema, welches gegenüber einer reinen Argumentenbilanz jedoch eine deutlich präzisere Entscheidung zulässt und die quantifizierbare Möglichkeit einer Wichtung bietet. Die Bewertung der einzelnen Konzepte erfolgte getrennt je Teilsystem des Aufbaus. Mithilfe des errechneten Gesamtwertes $G_V$ je Lösungsvariante aus den Maßzahlen $M$ und den Gewichtet $w$ konnte die Auswahl der besten Variante unmittelbar erfolgen~\cite{Pahl_Beitz_Konstruktionslehre}:

\begin{equation}
    G_{V_i} = \sum_{j=1}^{k} w_j \cdot M_{j,i}
    \label{eq:3_Gesamtwert_Variantenvergleich}
\end{equation}
\begin{equation}
    w_j \in \{\mathbb{Q} \;\vert\; 0 < w_j \leq 1\}; \qquad \sum_{j=1}^{k} w_j = 1; \qquad M_{j,i} \in \{1,\,2,\,3,\,4\}
    \label{eq:3_Wichtung_Bewertung}
\end{equation}
\begin{equation*}
    \text{\textit{k: Anzahl Kriterien; i: Variante; j: Kriterium}}
\end{equation*}

Eine höhere Maßzahl der Bewertung steht dabei stets für die jeweils bessere Variante innerhalb eines Kriteriums. Damit kann die beste Variante der jeweiligen Teilstruktur durch den höchsten Gesamtwert $G_V$ identifiziert werden. Die Wichtung erfolgte entsprechend der Kategorisierung und Wichtigkeit der jeweiligen Forderung.
\par
\vspace{\linespace}
Die Gestaltung der Probenhalterung als Teil der Messstrecke erfolgte ebenfalls im Rahmen der Konzeptphase. Die betrachteten Varianten wiesen jedoch nur geringe Unterschiede auf, sodass hier nur das gewählte Konzept als Teil des Entwurfes im \Abschnitt\ref{cha:3_Entwurf} beschrieben wird. Ähnliches gilt für die Auswahl geeigneter Absorberelemente zur Auskleidung des Innenraumes des Versuchsstandes und weitere Details, wie bspw. die Durchführungen der Antennenkabel, sodass diese und eine kurze Beschreibung der Entscheidungsgrundlage ebenfalls im \Abschnitt\ref{cha:3_Entwurf} vorgestellt werden.
%\par
%\vspace{\linespace}


\subsection{Schirmmodule des Versuchsstandes}\label{cha:3_sub_Schirmmodule_Versuchsstand}

Die Module bzw. Wände der Messkabine übernehmen die Schirmung äußerer Störeinflüsse und sind gleichzeitig Anschlussflächen zur Befestigung der Absorberelemente, Türen und Kabeldurchführungen. Die Bewertung der verschiedenen Wirkkonzepte erfolgte anhand der nachstehenden \acp{K}.

\begin{tabular}{l l}
    \hspace*{1cm} \parbox[c][3cm]{7cm}{
        \begin{itemize}[]
            \item \textbf{K\textsubscript{1}} Kosten
            \item \textbf{K\textsubscript{2}} Fertigungsaufwand
            \item \textbf{K\textsubscript{3}} Bedienerfreundlichkeit
        \end{itemize}
    }&
    \parbox[c]{7cm}{
        \begin{itemize}[]
            \item \textbf{K\textsubscript{4}} Gleichmäßigkeit des Kraftanstiegs
            \item \textbf{K\textsubscript{5}} Steuerung
            \item
        \end{itemize}
    }
\end{tabular}

In der \Tabelle\ref{tab:3_Module_Messkabine} ist die Konzeptbewertung und die Auswertung nach den \Gleichungen\eqref{eq:3_Gesamtwert_Variantenvergleich} und \eqref{eq:3_Wichtung_Bewertung} dargestellt.



\begin{table}[ht]
    \centering
    \renewcommand{\arraystretch}{1.2}
    \caption{Konzeptbewertung der Schirmmodule des Versuchsstandes}
    \vspace{\tablespace}
    \label{tab:3_Module_Messkabine}
    \begin{tabularx}{\textwidth}{p{4cm} r C{1cm} C{1cm} C{1cm} C{1cm} C{1cm} C{1.5cm}}
        \toprule
        \multirow{2}{*}{\textbf{Variante i}} & \textbf{Kriterien K\textsubscript{j}} & \textbf{K\textsubscript{1}} & \textbf{K\textsubscript{2}} & \textbf{K\textsubscript{3}} & \textbf{K\textsubscript{4}} & \textbf{K\textsubscript{5}} & \textbf{Summe} \\
         & Gewichte w\textsubscript{j} & 0,3 & 0,15 & 0,1 & 0,35 & 0,1 & \textbf{G\textsubscript{V}} \\
         \midrule
         \multicolumn{2}{l}{Variante 1 \hspace{0.6cm} \textit{Zugzylinder (elektrisch)}} & 2 & 1 & 3 & 3 & 3 & 2,40 \\
         \multicolumn{2}{l}{Variante 2 \hspace{0.6cm} \textit{Zugzylinder (händisch)}} & 3 & 1 & 2 & 2 & 4 & 2,35 \\
         \multicolumn{2}{l}{Variante 3 \hspace{0.6cm} \textit{Druckzylinder (elektrisch)}} & 2 & 2 & 3 & 3 & 3 & 2,55 \\
         \multicolumn{2}{l}{Variante 4 \hspace{0.6cm} \textit{Druckzylinder (händisch)}} & 4 & 2 & 2 & 2 & 4 & 2.80 \\
         \multicolumn{2}{l}{Variante 5 \hspace{0.6cm} \textit{Linearmotor}} & 1 & 3 & 4 & 4 & 1 & 2,65 \\
         \multicolumn{2}{l}{Variante 6 \hspace{0.6cm} \textit{Maschinenschraubstock}} & 3 & 4 & 1 & 1 & 4 & 2,35 \\
         \bottomrule
    \end{tabularx}
\end{table}



%Sandwichpaneele oder Wabenkernplatten mit Deckblechen <-- Wahl
%Mehrfachschirmung --> gleiche Schirmung bei geringerem Materialaufwand durch zusätzliche Reflektion innerhalb der Struktur (EMV-gerechtes Gerätedesign) + besseres Herabsetzen des Felddurchgriffes beim realen Schirm an Öffnungen, etc.  




\subsection{Durchführungen}\label{cha:3_sub_Durchfuehrungen}

Anhand der nachfolgenden Kriterien erfolgte die Bewertung der unterschiedlichen Varianten für die Türen des Versuchsstandes. Im Betrieb erlauben diese vor allem den Zugriff auf die Antennen und den Wechsel der Probekörper. Wie jede Öffnung in einer Schirmwand stellen sie neben den Kabeldurchführungen und den Anschlussstellen der Kammerwände eine der kritischsten Stellen in Bezug auf die erreichbare Schirmdämpfung der Messkabine dar (vgl. \Abschnitt\ref{cha:2_sub_Schirmung_ebener_Wellenfelder}). 

\begin{tabular}{l l}
    \hspace*{1cm} \parbox[c][3cm]{7cm}{
        \begin{itemize}[]
            \item \textbf{K\textsubscript{1}} Kosten
            \item \textbf{K\textsubscript{2}} Fertigungsaufwand
            \item \textbf{K\textsubscript{3}} Bedienerfreundlichkeit
        \end{itemize}
    }&
    \parbox[c]{7cm}{
        \begin{itemize}[]
            \item \textbf{K\textsubscript{4}} Gleichmäßigkeit des Kraftanstiegs
            \item \textbf{K\textsubscript{5}} Steuerung
            \item
        \end{itemize}
    }
\end{tabular}

In der \Tabelle\ref{tab:3_Durchfuehrungen} ist das Ergebnis der Konzeptbewertung ersichtlich.

\begin{table}[ht]
    \centering
    \renewcommand{\arraystretch}{1.2}
    \caption{Konzeptbewertung der Durchführungen des Versuchsstandes}
    \vspace{\tablespace}
    \label{tab:3_Durchfuehrungen}
    \begin{tabularx}{\textwidth}{p{4cm} r C{1cm} C{1cm} C{1cm} C{1cm} C{1cm} C{1.5cm}}
        \toprule
        \multirow{2}{*}{\textbf{Variante i}} & \textbf{Kriterien K\textsubscript{j}} & \textbf{K\textsubscript{1}} & \textbf{K\textsubscript{2}} & \textbf{K\textsubscript{3}} & \textbf{K\textsubscript{4}} & \textbf{K\textsubscript{5}} & \textbf{Summe} \\
         & Gewichte w\textsubscript{j} & 0,3 & 0,15 & 0,1 & 0,35 & 0,1 & \textbf{G\textsubscript{V}} \\
         \midrule
         \multicolumn{2}{l}{Variante 1 \hspace{0.6cm} \textit{Zugzylinder (elektrisch)}} & 2 & 1 & 3 & 3 & 3 & 2,40 \\
         \multicolumn{2}{l}{Variante 2 \hspace{0.6cm} \textit{Zugzylinder (händisch)}} & 3 & 1 & 2 & 2 & 4 & 2,35 \\
         \multicolumn{2}{l}{Variante 3 \hspace{0.6cm} \textit{Druckzylinder (elektrisch)}} & 2 & 2 & 3 & 3 & 3 & 2,55 \\
         \multicolumn{2}{l}{Variante 4 \hspace{0.6cm} \textit{Druckzylinder (händisch)}} & 4 & 2 & 2 & 2 & 4 & 2.80 \\
         \multicolumn{2}{l}{Variante 5 \hspace{0.6cm} \textit{Linearmotor}} & 1 & 3 & 4 & 4 & 1 & 2,65 \\
         \multicolumn{2}{l}{Variante 6 \hspace{0.6cm} \textit{Maschinenschraubstock}} & 3 & 4 & 1 & 1 & 4 & 2,35 \\
         \bottomrule
    \end{tabularx}
\end{table}



\section{Entwurf}\label{cha:3_Entwurf}



Im Anschluss an die erfolgte Konzeptphase und der Auswahl geeigneter Wirkkonzepte für die Teilfunktionen des Versuchsstandes werden die einzelnen Komponenten durch das Gestaltungskonzept einem Gesamtsystem zugeordnet. Der Entwurf legt dabei vorerst die grundlegende Gestaltung und Geometrie fest. Weiterhin werden strukturelle Elemente entwickelt, die zur Erfüllung der einzelnen Funktionen beitragen. An dieser Stelle sollen die Ergebnisse dieser iterativen Phase dargestellt werden.
\par
\vspace{\linespace}
Hauptdesigntreiber der äußeren Schirmhülle der Messkammer und damit der grundlegenden Struktur sind vor allem die Abmessungen. Die Auslegung soll in Anlehnung an geltende Normen, wie sie im \Abschnitt\ref{cha:2_sub_Genormte_Messverfahren} vorgestellt wurden, erfolgen. Des Weiteren fließen ebenfalls die Erkenntnisse aus dem \Abschnitt\ref{cha:2_sub_Feldverlauf_in_Umgebung_eines_Dipols}, in dem die verschiedenen Feldverlaufszonen in der Umgebung eines Dipols betrachtet wurden, in die Festlegung der Hauptmaße ein.
\par
\vspace{\linespace}
Problematisch für die Festlegung des Antennenabstandes zu den Probekörpern, um eine Messung im Fernfeld sicherzustellen, ist, dass der Beginn der Fraunhofer Region und damit des Fernfeldes in verschiedenen Veröffentlichungen teils sehr unterschiedlich abgeschätzt wird. Weiterhin sind gegebene Berechnungsvorschriften oft mit der Aussage verbunden, dass der Abstand deutlich größer als der berechnete Grenzwert sein sollte, wobei nicht näher spezifiziert wird, was deutlich größer in diesem Zusammenhang bedeutet. In der \Tabelle\ref{tab:3_Fernfeldabstaende} sind die unterschiedlichen Fernfeldgrenzen für eine Frequenz von \SI{1}{\giga\hertz} basierend auf den referenzierten Publikationen aufgeführt. Da die Wellenlänge mit steigender Frequenz abnimmt, wird nach \Abschnitt\ref{cha:2_sub_Feldverlauf_in_Umgebung_eines_Dipols} der Abstand des Fernfeldes zum Dipol ebenfalls kleiner. Damit stellt die Abschätzung für \SI{1}{\giga\hertz} den oberen Wert für die notwendige Ausdehnung der Messkammer dar.


\begin{table}[ht]
    \centering
    %\renewcommand{\arraystretch}{1.3}
    \caption{Fernfeldabstände für $f=1\;\si{\giga\hertz}$ auf Grundlage unterschiedlicher Veröffentlichungen}\label{tab:3_Fernfeldabstaende}
    \vspace{\tablespace}
    \begin{threeparttable}
    \begin{tabular}{@{\hspace{0.5cm}} p{4cm} R{1.3cm} @{,} p{0.5cm} @{m} p{1cm} p{0.5cm} p{1.5cm}}
    \toprule
        \textbf{Berechungsvorschrift} & \multicolumn{3}{c}{\textbf{Fernfeldabstand}} & \multicolumn{2}{c}{\textbf{Quelle}}  \\   %\footnotemark[1]
    \midrule
        $r >> \lambda / 2 \pi$  &     0&05  &&&  \cite{Klassische_Elektrodynamik} \\
        $r > 5 \lambda / 2 \pi$ &     0&24  &&&  \cite{EMV, EMV-gerechtes_Geraetedesign} \\
        $r \geq 2 D^2 / \lambda$&     0&27  &&&  \cite{Antenna_Theory}\footnotemark[1] \\
        $r > 4 \lambda$         &     1&2   &&&  \cite{Bundesnetzagentur_Glossar_Nahfeld} \\
        DIN EN 61000-5-7        &     2&0   &&& \cite{DIN_EN_61000-5-7} \\
        DIN EN 61000-4-3        &     1&0   &&& \cite{DIN_EN_61000-4-3}\footnotemark[2] \\
        IEEE 299                &     1&7   &&& \cite{IEEE_299} \\
        VG 95373-15             &     1&0   &&& \cite{VG_95373_15} \\

    \bottomrule
    \end{tabular}
    \begin{tablenotes}
    \footnotesize
    \item[1]Annahme: $D \approx 0,2\;\si{\meter}$ entsprechend der zu verwendenden Hornantennen
    \item[2]Mögliche Reduktion für $f\geq1\;\si{\giga\hertz}$
    \end{tablenotes}
    \end{threeparttable}
\end{table}


Für den Abstand der Proben zur Sendeantenne wurde auf Grundlage der Werte in \Tabelle\ref{tab:3_Fernfeldabstaende} ein Abstand von mindestens \SI{1}{\meter} gewählt. Dies ist nach den meisten Abschätzungen schon deutlich im Fernfeldbereich und in Anlehnung an die Normen~\cite{DIN_EN_61000-4-3, VG_95373_15}. Entsprechend der \Gleichungen\eqref{eq:2_elektrische_Feldvektoren} und~\eqref{eq:2_magnetische_Feldvektoren} sowie der \Abb\ref{fig:2_Feldwellenwiderstand} sind die Terme höherer Ordnung, die das reaktive und strahlende Nahfeld beschreiben, bei diesem Abstand und einer Frequenz von \SI{1}{\giga\hertz} vernachlässigbar und ihr Einfluss sinkt bei gleichem Abstand und steigender Frequenz noch weiter. 
\par
\vspace{\linespace}
Aus dem gewählten Abstand ergibt sich eine gesamte Messkammerlänge von etwa \SI{2}{\meter} unter Beachtung der Antennenausdehung und der Höhe gängiger Absorberelemente~\cite{Telemeter_Produktseite, EMV-Support_Produktseite}, die mindestens hinter beiden Antennen angebracht werden sollten. Bei einem Messausschnitt von $0,5 \times 0,5\;\si{\meter}$ in der Ebene der Prüflinge nach~\cite{DIN_EN_61000-4-3} ergibt sich eine Mindestbreite von circa \SI{1}{\meter} unter Beachtung der Absorber. Für die Höhe kann unter Berücksichtigung der Mindestabstandes der Prüflinge zum Boden von \SI{0,8}{\meter} nach~\cite{DIN_EN_61000-4-3, DIN_EN_61000-5-7} ein Wert von \SI{1,5}{\meter} inkl. Absorbern abgeschätzt werden. Um in beiden Ebenen orthogonal zur Ausbreitungsrichtung der Wellen ähnliche Feldeigenschaften an den Wänden der Messkabine zu erzielen, und damit ähnliche Bedingungen für alle Absorber, wurden \SI{1,5}{\meter} als Breite und Höhe der Testkammer gewählt. 
\par
\vspace{\linespace}
Die gewählten Maße bieten außerdem den Vorteil, dass keine Teilung der Modulwände in einer der Ausdehungsrichtungen erfolgen muss, da die meisten Anbieter von Sandwichpaneelen bzw. Wabenkernplatten in entsprechender Breite und Länge zur Verfügung haben. Dies reduziert die möglichen Stellen für Leckagen in der Schirmwand und den Fertigungsaufwand.
\par
\vspace{\linespace}
Die ersten Moden der Hohlraumresonanzfrequenzen liegen mit den Abmessungen nach \Gleichung\eqref{eq:2_Hohlraumresonanzfrequenz} im Bereich zwischen \SI{125}{\mega\hertz} und \SI{160}{\mega\hertz} und damit deutlich unter der kleinsten Messfrequenz. Die Ausbildung stehender Wellen höherer Ordnung muss dennoch durch Absorber unterdrückt werden.
\par
\vspace{\linespace}
Wie bereits erwähnt, erfolgt zur weiteren Reduktion der auftretenden Reflektionen und Resonanzen eine Auskleidung des Kammerinneren mit Absorberelementen. Am häufigstens finden dafür Absorber-Kacheln aus gesintertem Ferrit in Form von kleinen Fliesen und Pyramidenabsorber aus PU-Schaum Anwendung. Aufgrund der unterschiedlichen Wirkungsweise und Form sind diese jeweils für unterschiedliche Frequenzbereiche geeignet. Ferrite werden bis zu Frequenzen von \SI{1}{\giga\hertz} eingesetzt, während Pyramidenabsorber im Allgemeinen nur darüber Anwendung finden. Eine Kombination beider ist ebenfalls möglich, erfordert nach \Abschnitt\ref{cha:2_sub_Daempfung_und_Absorption} aber eine genaue Abstimmung, um keinen Impedanzsprung und damit eine reflektive Fläche zu schaffen. Aus Kostengründen und weil die Leistungsfähigkeit einzelner Pyradmidenabsorber auch im Bereich von \SI{1}{\giga\hertz} mit der von Kombinationen aus Ferrit und PU-Absorber vergleichbar ist, soll die Auskleidung mit einfachen Pyramidenelementen erfolgen. Die genaue Auswahl erfolgt im Rahmen der Ausarbeitung im folgenden Abschnitt. Die Höhe von Absorbern mit ausreichender Reflektionsverlustleistung beläuft sich auf etwa \SI{10}{\centi\meter} bis \SI{20}{\centi\meter}. Spezial- und Hochleistungsabsorber wurden für die vorliegende Anwendung nicht in Betracht gezogen. 
\par
\vspace{\linespace} 
Die Positionierung der Proben ist ebenfalls ein wichtiger Teil des Entwurfes. Die Probekörper sind nicht groß genug, um den Koppelpfad zwischen der Sende- und Empfangsantenne vollständig abzudecken. Dies macht einen Reflektor notwendigen, der den größten Teil der Wellen, die an den Proben vorbei in Richtung Empfangsantenne ausgesandt werden, reflektiert. So werden im Idealfall nur die Wellenanteile, die direkt durch die Proben verlaufen, von der Empfangsantenne aufgenommen. Die reflektierten Anteile verlieren ihre Energie an den Absorbern.   
\par
\vspace{\linespace}
Eine vollständige Trennung der Messkammer durch einen Reflektor ist aufgrund von entstehenden Resonanzen nach~\cite{Techniques_Shielding_Effectiveness_Far_Field_Simulation} nicht sinnvoll. Entsprechend erfolgt eine Einfügung des Reflektorschirms in den direkten Koppelpfad der Antennen mit einer Mindestausdehnung von $0,5\times$\SI{0.5}{\meter}~\cite{DIN_EN_61000-4-3}. Auf Grundlage der Betrachtungen im \Abschnitt\ref{cha:2_sub_Reflektion} kann als Reflektor eine Metallplatte mit möglichst hoher Leitfähigkeit genutzt werden.
\par
\vspace{\linespace} 
Das Wirkkonzept für den Einbau von Durchführungen wurde im \Abschnitt\ref{tab:3_Durchfuehrungen} bewertet. Bei einer gewählten Höhe der Messkammer von etwa \SI{1,5}{\meter} ist eine aufrecht begehbare Tür ohnehin nicht möglich, sodass aus Gründen des günstigeren Aufbaus das lichte Maß so gewählt wurde, dass eine Interaktion mit den Antennen und Probekörpern, bspw. zum Einbringen neuer Proben, gut möglich ist. Das Betreten des Teststandes sollte nur in Ausnahmefällen bzw. im Rahmen des Aufbaus möglich sein. Bezüglich der erreichbaren Schirmdämpfung spielen das lichte Maß oder das Türblattaußenmaß nur dahingehend eine praktische Rolle, als dass auf der gesamten Wirkfläche ein möglichst gleichmäßiger Anpressdruck an die HF-Dichtungen erreicht wird. Mit der Größe des Türblattes steigt die Anzahl notwendiger Haltepunkte bzw. das notwendige Flächenträgheitsmoment zum Gewährleisten einer ebenen Anschlussfläche an den HF-Dichtungen. Weiterhin müssen mit steigendem Gewicht des Türblattes natürlich auch alle Befestigungen massiver ausgeführt werden. Als Kompromiss zwischen diesen drei Optimierungszielen wurden $700 \times 700\;\si{\milli\meter}$ als lichtes Maß gewählt. Um Zugang sowohl zur Sende- als auch zur Empfangsantenne von außen zu gewährleisten, werden in der Front der Messkammer zwei Türen vorgesehen.
\par
\vspace{\linespace}
Die im Konzept der Durchführungen enthaltenen Kontaktfederstreifen bieten die beste erreichbare Schirmdämpfung im Vergleich zu anderen HF-Dichtungstypen~\cite{EM_Schirmung}. Gleichzeitig besitzen sie selbstreinigende Eigenschaften, denn aufgrund der Relativbewegung zwischen Rahmen und Dichtung werden Oxid- und andere Oberflächenschichten bei jedem Schließvorgang entfernt und es entsteht ein sauberer Kontakt mit geringem Kontaktwiderstand. Im Rahmen dieses Projektes sind die hohen Kosten im Vergleich zu anderen Dichtungen der hauptsächliche Nachteil, was jedoch durch die hohe Beständigkeit teilweise relativiert wird. Um einen guten Kontakt sicherzustellen, sind bei den geplanten Beryllium-Kupfer-Kontaktferderstreifen höhere Anpresskräfte notwendig, als bpsw. bei Elastomerdichtungen.  
\par
\vspace{\linespace}
Für die Durchführungen von Leitungen durch die Schirmwand und in die Messkammer gibt es mehrere Möglichkeiten. Die einfachste ist die Nutzung einer Eigenschaft von Hohlwellenleitern (vgl. \Abschnitt\ref{cha:2_subsub_Hohlwellenleiter}), der sogenannten Cut-off-Frequenz oder kritische Frequenz $f_k$ mit der zugehörigen kritischen Wellenlänge $\lambda_k$. Unterhalb dieser Frequenz ist in einem Hohlwellenleiter gegebener Geometrie keine Wellenausbreitung möglich und es findet eine aperiodische Dämpfung statt. Bei der Nutzung als Durchführung kann aus der höchsten Einsatzfrequenz eines Schirms die notwendige Geometrie abgeschätzt werden. Nach \Tabelle\ref{tab:2_Grenzwellenlaengen_Hohlleiter} ergibt sich für einen Kreisquerschnitt der größtmögliche Durchmesser für ein Kabel, was in den Schirm geführt werden soll. Bei \SI{18}{\giga\hertz} als höchste Einsatzfrequenz ergibt sich

\begin{equation}
    d(18\;\si{\giga\hertz}) \approx \frac{\lambda(18\;\si{\giga\hertz)}}{1,305} = \frac{c_0}{18\;\si{\giga\hertz}\cdot 1,305} \approx 0,0127 \; \si{\meter}
\end{equation}

als größter Durchmesser für eine Kabeldurchführung mittels Hohlwellenleiter. Unter Beachtung, dass die kritische Frequenz im Anwendungsfall noch deutlich unter der höchsten Einsatzfrequenz liegen sollte~\cite{EM_Schirmung}, eignet sich diese Methode aufgrund des geringen möglichen Kabeldurchmessers nicht zur Durchführung der geschirmten Antennenkabel vom \ac{VNA} zu den Antennen. Hinzu kommt, dass eine niederohmige Verbindung des Kabelschirmes mit der Schirmwand, wie sie bei der Durchführung geschirmter Kabel notwendig ist~\cite{EM_Schirmung, EMV}, in diesem Fall fast nur durch Anlöten des Kabelschirmes zu erreichen wäre.
\par
\vspace{\linespace}
Am besten geeignet für die Einführung geschirmter HF-Kabel in ein Gehäuse sind Konnektoren in der Art, wie sie auch zum Verbinden der Kabel mit dem \ac{VNA} genutzt werden. Dies bietet außerdem den Vorteil, dass die Antennenkabel unabhängig vom \ac{VNA} stets in der Messkammer verbleiben können und der \ac{VNA} durch schnelles Lösen der Aus- und Eingangskabel mit wenigen Handgriffen anderweitig genutzt werden kann. Für Schnittstellen an der Messkabine wird u.a. in~\cite{EM_Schirmung, EMV} die Verwendung einer Kupferplatte empfohlen, die aufgrund des sehr geringen Widerstandes Störströme gut ableiten kann. Dies bietet weiterhin den Vorteil, dass neue Schnittstellen direkt in der Kupferplatte hinzugefügt werden können, ohne dass die komplette Schirmwand nochmals bearbeitet werden muss. Von der Durchführung von Kabeln an unterschiedlichen Stellen des Schirms wird in~\cite{EM_Schirmung, EMV, Design_of_shielded_enclosures} ausdrücklich abgeraten. 
\par
\vspace{\linespace}
Um bei den gewählten Abmessungen einen guten Zugang zu den Öffnungen zu gewährleisten, wird ein Untergestell aus Aluminiumprofilen genutzt, welches den Teststand auf eine gut erreichbare Höhe hebt.
\par
\vspace{\linespace}
In der \Abb\ref{fig:3_Entwurf_CAD} ist das Gestaltungskonzept mit einer Ansicht des CAD-Modells veranschaulicht. 


\begin{figure}
    \centering
    %\includegraphics{}
    \caption{CAD-Modellansicht des Versuchsstandes}
    \label{fig:3_Entwurf_CAD}
\end{figure}



%Türen

%Dichtungen

%Kabeldurchführungen

%Gestell

%Beleuchtung












\section{Ausarbeitung}\label{cha:3_Ausarbeitung}


Die Detaillierung des Entwurfes basiert größtenteils auf Maßen verfügbarer Kaufteile und Abschätzungen für gewählte Abmessungen anhand von Ausführungen oder Beispielen in den herangezogenen Veröffentlichungen. Dabei sind vor allem~\cite{Design_of_shielded_enclosures, EMV-gerechtes_Geraetedesign, EM_Schirmung, Simplified_shielding, Handbook_Shielding_Materials_and_Performance} zu nennen, die als Referenz für Abschätzungen dienten. Die folgenden Ausführungen schildern das Ergebnis des Konstruktionsprozesses in einer möglichst systematischen Reihenfolge.

\subsection{Verwendete Messtechnik}

Die Hauptkomponente der verwendeten Messtechnik stellt, wie bereits erwähnt, ein \ac{VNA} dar. Der für diese Arbeit zur Verfügung stehende \ac{VNA} der Firma Anritsu ist ein Zweikanal-Netzwerkanalysator mit einem Arbeitsbereich zwischen \SI{1}{\mega\hertz} und \SI{20}{\giga\hertz}. Ein Auszug mit den für dieses Modell relevanten Daten und Diagrammen nach~\cite{VNA-Datenblatt} ist im \Anhang\ref{A:Datenblatt_VNA} zu finden. 
\par
\vspace{\linespace}
Da der \ac{VNA} mithilfe der zwei verfügbaren Ports sowohl die Sendeantenne speist, als auch das Messsignal an der Empfangsanatenne aufnimmt, kann die Ermittlung der Schirmdämpfung direkt von der zugehörigen Software erfolgen. Außerdem sind sogenannte Zeitbereichsmessungen (Time Domain Measurements) möglich, wodurch eine Unterscheidung zwischen dem direkten und indirekten Koppelpfad möglich ist~\cite{Techniques_Shielding_Effectiveness_Far_Field_Simulation}. Im \Kapitel\ref{cha:4} wird darauf näher eingegangen.
\par
\vspace{\linespace}
Die Kalibration der Messtechnik erfolgt mithilfe des zugehörigen Kalibrationskits, wodurch bei korrekter Durchführung die im Auszug des Datenblattes (vgl. \Anhang\ref{A:Datenblatt_VNA}) dargestellten Unsicherheiten eingehalten werden sollten. Die durchgeführte Kalibration des \ac{VNA} wird im \Abschnitt\ref{cha:4_Kalibration_Messtechnik} ausführlich dargestellt.
\par
\vspace{\linespace}
Der \ac{VNA} soll mittels der ebenfalls zugehörigen Signalkabel, die eine Wellenimpedanz von \SI{50}{\ohm} aufweisen, mit den Antennen verbunden werden~\cite{Testkabel_VNA-Datenblatt}. Die Anschlüsse sind \ac{SMA} kompatible Steckverbinder mit \SI{2,92}{\milli\meter}-Stecker.
\par
\vspace{\linespace}
%Antennen/Waveguides
    %Größe
    %Platziert mit vorhandenen Stativen
Die zur Verfügung stehenden Antennen sind Breitband-Hornstrahler mit einer linearen Polarisation und den in der \Tabelle\ref{tab:3_Spezifikationen_Antennen} zusammengefassten wichtigstens Spezifikationen nach~\cite{Antennen-Datenblatt}. Weitere Kenndaten sind im \Anhang\ref{A:Datenblatt_Antennen} zu finden. Die Antennen bieten ebenfalls einen SMA-Anschluss für die Signalkabel.

\begin{table}[ht]
    \centering
    \caption{Technische Spezifikationen der verwendeten Hornstrahler nach~\cite{Antennen-Datenblatt}}
    \label{tab:3_Spezifikationen_Antennen}
    \begin{tabular}{p{6cm} C{5cm}}
    \toprule
        \textbf{Spezifikation} & \textbf{Wert} \\
    \midrule
        Frequenzbereich & $0,8 - 18\;\si{\giga\hertz}$ \\
        Halbwerts-Strahlbreite  & $111-13 \;\si{\degree}$ (E-Feld) \\
                                & $78-10\;\si{\degree}$ (H-Feld) \\
        Kreuzpolarisation-Isolation & \SI{25}{\Dezibel} (typ.) \\
        Stehwellenverhältnis (VSWR) & $1,5 : 1$ (typ.) \\
        Abmessungen             & $244\times160,5\times228\;\si{\milli\meter}$ \\
    \bottomrule
    \end{tabular}
\end{table}

Mithilfe der zugehörigen Aluminium-Stative können die Antennen frei auf einer Höhe zwischen \SI{53}{\centi\meter} und \SI{158}{\centi\meter} platziert werden.




\subsection{Konstruktion}

Die Modulwände werden hauptsächlich aus den Sandwichpaneelen gebildet, wodurch die Auswahl dieser von zentraler Bedeutung für die restliche Konstruktion ist. Aufgrund von Produktverfügbarkeiten mussten Verbundplatten aus Sperrholz mit metallischen Deckblechen, wie sie im Rahmen professioneller Absorberkammern zum Einsatz kommen, schon zu Beginn der Konzeptphase verworfen werden. Als Alternative wurden Wabenkernplatten gewählt, die ebenfalls durchgehende Deckbleche besitzen, aufgrund der inneren Struktur ein deutlich günstigeres Verhältnis aus Biegesteifigkeit und Gewicht besitzen und auf ganz ähnliche Weise verarbeitet werden können~\cite{Alucore-Datenblatt}. Im Messbereich zwischen $1\ldots18\;\si{\giga\hertz}$ ist nach \Abschnitt\ref{cha:2_sub_Schirmung_ebener_Wellenfelder} die Absorptions- und Reflektionsdämpfung für die kombinierten \SI{2}{\milli\meter} Blechstärke bereits so hoch, dass für die real erreichte Schirmung des Versuchsstandes gegenüber Störquellen von außen nur noch Öffnungen wie Türen oder Verbindungsstellen entscheident sind.  
\par
\vspace{\linespace}
Für den vorgesehenen Einsatzzweck musste die Polyesterlackschicht an allen Wirkflächen, das heißt allen Verbindungsflächen der Modulwände untereinander und allen weiteren Anschlussflächen, mechanisch entfernt werden. Um trotz der sich stets an Luftatmosphäre bildenden Oxidschichten einen leitfähigen Kontakt an allen Wirkflächen herzustellen, werden Funktionselemente nach Möglichkeit miteinander verschraubt. Dabei kann im Fall von Aluminium bereits bei kleinen Vorspannkräften im Bereich von \SI{1000}{\newton} davon ausgegangen werden, dass die harten Oxidschichten aufreißen und Metall-Metall-Kontakt entsteht~\cite{Projektarbeit}. Im Fall der Türdichtungen gewährleisten die gewählten HF-Dichtungen aufgrund der Relativbewegung bei jedem Schließvorgang eine mechanische Entfernung der Oxidschichten.
\par
\vspace{\linespace}
Die Grundlage für die Auslegung der L-Profile bildet der aus der Schraubenberechnung bekannte Rötscher-Kegel. Die Schenkellänge der Profile, welche die Wabenkernplatten an den Eckstößen miteinander verbinden, wurden entsprechend der verfügbaren Profilmaße so gewählt, dass sich der Verspannungskegel innerhalb der verschraubten Teile vollständig ausbilden kann und somit die Vorspannkraft der Schrauben möglichst großflächig in die Wabenkernplatte eingeleitet wird. Die \Abb\ref{fig:3_Verspannungskegel_L-Profile} zeigt dies in einer schematischen Darstellung zusammen mit den gewählten Profilmaßen und dem sich ergebenden Ersatzquerschnitt des Schraubverbandes. 

\begin{figure}[ht]
    \centering
    \includegraphics[page=1, trim=0cm 0cm 0cm 0cm, clip, width = .45\textwidth]{Abbildungen/Kapitel3/Schematik_Verspannungskegel.pdf}
    \caption{Schematische Darstellung des Verspannungsbereiches der verschraubten Modulwände}
    \label{fig:3_Verspannungskegel_L-Profile}
\end{figure}


Die Wahl des Schraubenabstandes wurde auf Grundlage eines Zusammenhanges nach~\cite{Design_of_shielded_enclosures} getroffen, welcher die erreichbare Schirmdämpfung als Funktion des Schraubenabstandes verschraubter Blechteile darstellt. Wie aus der Grafik in \Abb\ref{fig:3_Schirmwirkung_Schraubenabstand} hervorgeht, kann bei einem Schraubenabstand von \SI{10}{\centi\meter} von etwa \SI{70}{\Dezibel} theoretisch erreichbarer Schirmdämpfung je Funktionsfläche ausgegangen werden. Dieser Abstand wurde auch in Hinblick auf den Fertigungsaufwand der Bohrungen und Verschraubungen gewählt. Aufgrund der Sandwichbauweise der Modulwände befinden sich des Weiteren stets zwei Wirkflächen im Koppelpfad, wodurch der Durchgriff weiter verringert wird. Die Anfertigung der Bohrungen erfolgte händisch und während des Zusammenbauprozesses, um eine möglichst exakte Ausrichtung der Bohrlöcher in den einzelnen verspannten Teilen relativ zueinander zu erreichen.  


\begin{figure}[ht]
    \centering
    \includegraphics[page = 1, trim = 0cm 0cm 0cm 0cm, clip, width=.65\textwidth]{Abbildungen/Kapitel3/Schraubenabstand_Schirmwirkung.pdf}
    \caption[Schirmdämpfung verschraubter Blechteile in Abhängigkeit des Schraubenabstandes]{Schirmdämpfung verschraubter Blechteile in Abhängigkeit des Schraubenabstandes nach~\cite{Design_of_shielded_enclosures}}
    \label{fig:3_Schirmwirkung_Schraubenabstand}
\end{figure}

\par
\vspace{\linespace}
Um eine möglichst hohe Schirmdämpfung zu erreichen, wurde die Anordnung der Profile mit entsprechenden Aussparungen so gewählt, dass sich an keiner Stelle zwei Anschlussstellen direkt gegenüberstehen. Dies führt zu einer Labyrinthwirkung an Stellen von Profilübergängen und verringert somit ebenfalls den Felddruchgriff an den Kontaktstellen der Profile. Mit dem gewählten Konzept wäre eine höhere Schirmwirkung an den Verbindungsstellen der Profile untereinander nur durch dichtes Verschweißen oder Verlöten zu erreichen~\cite{Design_of_shielded_enclosures, EM_Schirmung}. 
\par
\vspace{\linespace}


%Wabenkernplatten als Grundlage 
    %Verfügbarkeit
    %Vorteile gegenüber Vollmaterial (ggf. mit Grafik bzw. Wert
    %Datenblatt ggf. einfügen
    %Besonderheiten (Entfernen der Lackschicht z.B.)
    
%Aluprofile mit Rechnung bzw. der gewählten Dicke und Maßen

%Schraubenberechnung ggf.

%(Aufbau-(Reigenfolge))




Die Auswahl geeigneter Absorberelemente zur Reduktion von Reflektionen und Vermeidung von Hohlraumresonanzen in der Messkabine wurde unter Einbeziehung der erreichbaren Reflektionsdämpfung, der Höhe einzelner Elemente und ökonomischen Gesichtspunkten getroffen. Die Reflektionsdämpfung gibt hierbei das Verhältnis einer eintreffenden Welle zum reflektierten Anteil an. Eine möglichst hohe und gleichbleibende Reflektionsdämpfung über alle Einsatzfrequenzen ist dementsprechend wünschenswert. Im \Abschnitt\ref{cha:3_Entwurf} wurde die Auswahl bereits auf reine Pyramidenabsorber eingeschränkt. Zur Vermeidung von reflektivem Verhalten bei hohen Frequenzen wurde auf Absorberelemente mit abgeschnittener Spitze verzichtet. 
\par
\vspace{\linespace}
Die gewählten Absorber besitzen die in \Tabelle\ref{tab:3_Reflektionsdaempfung_Absorberelemente} angegebene garantierte Reflektionsdämpfung im Messbereich dieser Arbeit. Mit einer Höhe von \SI{10}{\centi\meter}  verbleibt auch nach dem Einbau noch ein großer Messbereich in der Testkammer. Des Weiteren ist vor allem im höheren Frequenzbereich eine höhere Reflektionsdämpfung zu erwarten, als bei der \SI{20}{\centi\meter} hohen Variante der gleichen Produktfamilie. Auf die Verwendung noch höherer Elemente wurde aufgrund der steigenden Einschränkung des Messraumes und ökonomischer Sicht verzichtet.  


\begin{table}[ht]
    \centering
    \begin{tabular}{p{2cm} C{1.6cm} C{1.6cm} C{1.6cm} C{1.6cm} C{1.6cm}}
        \toprule
            &   \multicolumn{5}{l}{\textbf{Reflektionsdämpfung [\SI{}{\Dezibel}]}} \\
        \midrule
            &   \SI{1}{\giga\hertz} & \SI{3}{\giga\hertz} & \SI{5}{\giga\hertz} & \SI{10}{\giga\hertz} & \SI{18}{\giga\hertz} \\
        \textbf{Gemessen}   &   15  &   32  &   42  &   50  &   55 \\    
        \textbf{Garantiert} &   12  &   30  &   40  &   45  &   50 \\
        \bottomrule
    \end{tabular}
    \caption[Gemessene und garantierte Reflektionsdämpfung der verwendeten EPP12 Pyramidenabsorber im Bereich zwischen \SI{1}{\giga\hertz} bis \SI{18}{\giga\hertz}]{Gemessene und garantierte Reflektionsdämpfung der verwendeten EPP12 Pyramidenabsorber im Bereich zwischen \SI{1}{\giga\hertz} bis \SI{18}{\giga\hertz} nach~\cite{Eco_Messtechnik_Absorber}}
    \label{tab:3_Reflektionsdaempfung_Absorberelemente}
\end{table}

Auf dem Messprotokoll~\cite{Eco_Messtechnik_Absorber} sind außerdem keine ausgezeichneten Peaks des Dämpfungsverhaltens zu erkennen. In Bezug auf Kosten und Höhe vergleichbare alternative Absorber bieten ab etwa \SI{5}{\giga\hertz} nur eine um \SI{5}{\Dezibel} reduzierte Reflektionsdämpfung, was bei Betrachtung des Verhältnisses der Feldgrößen etwa einem Faktor 2 entspricht (vgl. \Tabelle\ref{tab:2_Relative_Pegel})~\cite{Holland_Shielding_Absorber}. Mit Ferritkacheln und zusätzlich darauf abgestimmten Hybridabsorbern kann zwar bereits im Frequenzbereich zwischen \SI{1}{\giga\hertz} bis \SI{3}{\giga\hertz} eine Reflektionsdämpfung von \SI{25}{\Dezibel} erreicht werden, jedoch ist mit dieser Kombination aufgrund der Impedanzabstimmung der PU-Absorber auf die Ferritkacheln bei hohen Frequenzen nur mit etwas mehr als \SI{30}{\Dezibel} zu rechnen, ab \SI{10}{\giga\hertz} sogar nur mit weniger als \SI{30}{\Dezibel}~\cite{Holland_Shielding_Absorber}, mit teils stark ausgeprägten Peaks in bestimmten Frequenzbändern. Aus diesen Gründen wurden die oben beschriebenen Absorber des Types EPP12 der Firma \Firma{ECO-Messtechnik GmbH und Co. KG} verwendet, die außerdem nach DIN 4102 als nicht brennbar der Klasse M2 (CSTB) eingestuft wurden~\cite{Eco_Messtechnik_Absorber}. 
\par
\vspace{\linespace}
Da aufgrund der Höhe der Pyramidenabsorber keine exakte Auskleidung der Schnittstellen zwischen den Modulwänden erfolgen kann, wurden zusätzliche Flachabsorber des Types EPF15, welche ebenfalls von \Firma{ECO-Messtechnik GmbH und Co. KG} angeboten werden~\cite{Eco_Messtechnik_Absorber}, an den Ecken des Versuchsstandes angebracht. Die Befestigung der Absorberelemente erfolgte mittels Klettbändern, um eine nachträgliche Justierung oder einen Austausch zu ermöglichen.   
\par
\vspace{\linespace}

%Absorberelementauswahl 
    %Datenblatt
    %Anzahl Abschätzung
    %Flachabsorber für Ecken
    %Befestigung
    




    
%Probenreflektor mit Probenhalterung (CAD ggf.) und Beschreibung der Probenhalterung mit Vorteil der verschiedenen Dicken der Probekörper

%ggf. Prositionierung der Antennen (Stativ aus Holz)

%Türen mit Verschlüssen
    %Begründung Anzahl Verschlüsse
    %Detail mit Abstandshalter der Scharniere (damit Kontaktfederstreifen nicht zusammengedrückt werden)

%Auswahl Kabeldurchführungen (Impedanz von 50 Ohm im Match mit VNA --> Verweis Datenblatt VNA und Kabeldurchführungen (Muss matchen, um keine Rückreflektion zu bekommen (siehe Folien von Connector Care))
    %ggf. Detail mit Kupferplatte hierhin und nicht in Entwurf
    




