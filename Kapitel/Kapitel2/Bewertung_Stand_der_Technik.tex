
Trotz umfangreicher Fachliteratur zum Themenkomplex der Elektromagnetischen Verträglichkeit und den Möglichkeiten der Reduktion von Strahlungseinflüssen auf elektronische Baugruppen, gibt es nur vergelichsweise wenige Autoren, die sich intensiv mit der theoretischen Beschreibung der Schirmwirkung von Materialien oder der experimentellen Bestimmung auseinandersetzen. Dies gilt vor allem für den Frequenzbereich ab \SI{1}{\giga\hertz}. Standardisierte Messverfahren zur Bestimmung der Schirmdämpfung in einer simulierten Fernfeldumgebung sind nur bis etwa \SI{1,6}{\giga\hertz} anwendbar (vgl. \Abschnitt\ref{cha:2_sub_Genormte_Messverfahren}), darüber hinaus sind alle genormten Verfahren nicht direkt für Materialproben anwendbar. 
\par
\vspace{\linespace}
In publizierten Artikeln werden zum Teil sehr verschiedene Verfahren zur Schirmdämpfungsmessung im Fernfeld für den Hochfrequenzbereich angewandt. Erwähnt seien an dieser Stelle die Veröffentlichungen~\cite{Measurement_Shielding_Textile_Materials_Free_Space_Transmission, Predicted_and_Measured_EMI_Effectiveness_Metallic_Mesh_on_Window}, deren Messungen ebenfalls an~\cite{DIN_EN_61000-5-7, IEEE_299} angelehnt sind. Der Aufbau der verwendeten Absorberkammer wurde dabei in keiner der untersuchten Publikationen thematisiert. Weiterhin sind die verwendeten Materialproben fast ausschließlich um ein Vielfaches größer als die CNT-Folien, die im Rahmen dieser Arbeit untersucht werden sollen.
\par
\vspace{\linespace}
Die betrachteten theoretischen Grundlagen der Eigenschaften von Hochfrequenzfeldern sind essentiell für die Kontruktion des verwendeten Messtandes. Dies gilt sowohl für die möglichst hohe Störfestigkeit der Messumgebung, als auch für die Einstellung geeigneter Feldeigenschaften im Messbereich. Im folgenden \Kapitel\ref{cha:3} wird darauf näher eingegangen.



%In~\cite{Measurement_Shielding_Textile_Materials_Free_Space_Transmission, Predicted_and_Measured_EMI_Effectiveness_Metallic_Mesh_on_Window} wurden ebenfalls an diese Messverfahren angelehnte Methoden zur Bestimmung der Schirmdämpfung von Materialien im Frequenzenbereich zwischen \SI{1}{\giga\hertz} bis mindestens \SI{18}{\giga\hertz} genutzt. Der Aufbau der Schirmkammer zur Durchführung der Messungen wurde in keiner der betrachteten Arbeiten thematisiert.

%In keiner Arbeit so kleine Probekörper genutzt
%Normen nur bis 1,6 GHz zur Messung der Materialeigenschaft in Fernfeldsimulation


%ggf. auf Einleitung eingehen