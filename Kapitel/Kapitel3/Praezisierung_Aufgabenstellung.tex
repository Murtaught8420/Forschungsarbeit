


Im Verlauf der Aufgabenanalyse erfolgt die Erstellung einer Anforderungsliste, welche die Forderungen an die Konstruktion möglichst umfangreich zusammenstellt. Die Anforderungen setzen sich hauptsächlich aus den Rahmenbedingungen und Wünschen hinsichtlich verschiedener Hauptmerkmale und zur Funktionserfüllung notwendigen Forderungen zusammen. In Anlehnung an die Kategorisierung, wie sie in der Literatur beschrieben wird, wurde die Unterteilung in \acp{F} und \acp{W} vorgenommen. Dadurch konnten Anforderungen identifiziert werden, die durch alle Konzepte unbedingt zu erfüllen sind und solche, die nach Möglichkeit Berücksichtigung finden sollten. Der Erfüllungsgrad aller Anforderungen bildete die Bewertungsgrundlage der Konzeptideen.
\par
\vspace{\linespace}
Die Präzisierung der Detailanforderungen erfolgte teilweise und insofern möglich in Iterationsschleifen während der Konzept- und Entwurfsphase, wobei hier nur das Ergebnis dargestellt werden soll. Die Anforderungsliste im \Anhang\ref{A:Anforderungsliste} enthält die  soweit möglich konkretisierten Anforderungen sowie zusätzlich und falls notwendig eine Bemerkung zur Hauptquelle für die Präzisierung.
\par
\vspace{\linespace}
Der Fokus der Entwicklung soll vor allem auf einer hohen Wiederholbarkeit der Messungen liegen, da eine Abschätzung der Reproduzierbarkeit aufgrund von geometrischen Abweichungen und unterschiedlichen Antennencharakteristika im Voraus so gut wie nicht möglich ist (vgl. \Abschnitt\ref{cha:2_Methoden_der_Schirmdaempfungsmessung}).





