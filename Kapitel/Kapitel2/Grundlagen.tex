
Allgemein beschreibt ein Feld die Gesamtmenge aller Werte einer physikalischen Größe, die allen Punkten eines leeren oder stoffgefüllten Raumes zugeordnet sind, die sogenannte Feldgröße \cite{Spektrum.de_Feld}. Je nach Art der Feldgröße wird der allgemeine Feldbegriff in Skalarfeld 

\begin{equation}
    a(x,y,z, \ldots)    
\end{equation}


und Vektorfeld 

\begin{equation}
    \vec A(x,y,z,\ldots)
\end{equation}

unterteilt. Durch die Darstellung der Abhängigkeit von den Ortskoordinaten des Raumes wird deutlich, dass es sich bei einer Größe um ein Feld handelt. \par
\vspace{\linespace}
Die im Rahmen dieser Arbeit wichtigen Felder sind das elektrische und das magnetische Feld. Eine Beschreibung des Zusammenhanges zwischen der Feldursache und dem entstehenden Feld lässt sich in beiden Fällen mithilfe der Materialgleichungen und der Maxwell'schen Gleichungen vornehmen~\cite{EM_Schirmung}. Grundlegend lässt sich ein Feld jedoch auch ohne Kenntnis seiner Ursache beschreiben, sodass im Rahmen dieser Arbeit nicht explizit auf die Maxwell'schen Gleichungen eingegangen werden soll, da dies thematisch zu weit führen würde. 


\subsection{Elektrisches Feld}\label{cha:2_sub_Elektrisches_Feld}

Um das elektrische Feld mittels seiner Wirkungs zu beschreiben, wird eine Testladung in das Feld eingebracht, auf die daraufhin eine Kraft $\vec F_q(x,y,z,q)$ wirkt. Da die Kraft eine gerichtete Größe ist, muss die Beschreibung des elektrischen Feldes mittels eines Vektorfeldes beschrieben werden. Die Größe der Kraft ist abhängig von der eingebrachten Ladung. Um eine Veränderung des Feldes durch die Testladung auszuschließen, wird stattdessen eine infinitesimale Teilladung $dq$ zur Beschreibung des Feldes verwendet, was sich damit aus

\begin{equation}
    \vec E (x,y,z) = \frac{d\vec F_q(x,y,z,q)}{dq}
\end{equation}

ergibt. Der Differentialquotient $\vec E$ ist die elektrische Feldstärke \cite{EM_Schirmung}. \par
\vspace{\linespace}
Damit ist die Wirkung eines elektrischen Feldes beschrieben, jedoch noch nicht dessen Ursache. Nach dem sogenannten Satz des Hüllenflusses, dem ersten Maxwell'schen Gesetz, erfahren Ladungen nicht nur Beeinflussung durch ein elektrisches Feld, sondern sind auch dessen Ursache. Eine Beschreibung kann am einfachsten mithilfe eines Plattenkondensators erfolgen, dessen Plattenflächen $A_P$ senkrecht auf den Feldlinien eines elektrisches Feldes stehen und auf welche die Ladungen $+q$ und $-q$ genau so aufgebracht werden, dass im Inneren des Kondensators das äußere Feld kompensiert wird. Eine von der Fläche unabhängige Größe zur Beschreibung des elektrischen Feldes lässt sich wiederum durch den Differentielquotienten 

\begin{equation}
    \vec D(x,y,z) = \frac{dq}{dA_P} \cdot \vec n_A
\end{equation}

erhalten, der elektrischen Flussdichte $\vec D$ \cite{EM_Schirmung}. \par
\vspace{\linespace}

Da sowohl die Feldstärke als auch die Flussdichte das elektrische Feld beschreiben, gibt es in Abhängigkeit Materials, aus dem der felderfüllte Raum besteht, einen Zusammenhang zwischen beiden Größen, die sogenannte Dielektrizitätszahl oder Permittivität $\varepsilon$, die eine Materialeigenschaft ist. Es gilt

\begin{equation}
    \vec D = \varepsilon \cdot \vec E = \varepsilon_0 \varepsilon_r \cdot \vec E \qquad \quad \text{mit} \qquad \varepsilon_0 = 8,85419 \cdot 10^{-12} \; \si{\ampere\second\per\volt\per\meter}
\end{equation}

mit der Dielektrizitätszahl des leeren Raumes $\varepsilon_0$ und der relativen Dielektrizitätszahl $\varepsilon_r$ des betrachteten Materials \cite{EM_Schirmung}.


\subsection{Magnetisches Feld}\label{cha:2_sub_Magnetisches_Feld}

Wie auch die elektrische Feldstärke wird die magnetische Flussdichte indirekt beschrieben, d.h. über ihre messbare Kraftwirkung auf elektrische Ströme bzw. bewegte Ladungen. Dabei wird ein stromdurchflossener Draht der Länge $L$ so über einem Magnetfeld angeordnet, dass die messbare Kraft maximal wird. Um auch hier Rückwirkungen auszuschließen, wird eine differentielle Drahtlänge, die vom infinitesimal kleinen Strom $dI$ durchflossen wird, betrachtet. Mit des Differentialquotienten

\begin{equation}
    \vec B(x,y,z) = \frac{d^2 \vec F_L(x,y,z,L,I)}{dL \cdot dI}
\end{equation}

lässt sich die magentische Flussdichte, die sowohl vom Strom $I$ als auch von der Drahtlänge $L$ abhängt, aus der gemessenen Kraft ermitteln \cite{EM_Schirmung}. 
\par
\vspace{\linespace}
Analog zur Betrachtung elektrischer Felder lässt sich feststellen, dass Ströme nicht nur eine Kraftwirkung durch magnetische Felder erfahren, sondern auch deren Ursache sind. Dazu lässt sich ebenfalls ein einfaches Gedankenexperiment ähnlich der Betrachtung zur elektrischen Flussdichte durchführen: 
\par
\vspace{\linespace}
In einer Spule der Länge $L$ fließe ein Strom $I$, der genau so groß ist, dass durch die induzierte magnetische Flussdichte ein umgebendes äußeres Magnetfeld im Inneren der Spule verschwindet. Die Ausrichtung der Anordnung sei wiederum so erfolgt, dass der Strom $I$ maximal wird. Der erforderliche Strom ist abhängig von der Spulenlänge und deren Windungszahl $N_w$. Das Differential

\begin{equation}
    \vec H(x,y,z) = N_w \frac{dI(x,y,z,L)}{dL} \cdot \vec n_A
\end{equation}

beschreibt die magnetische Feldstärke, welche senkrecht auf der Spulenquerschnittsfläche steht \cite{EM_Schirmung}. 
\par
\vspace{\linespace}
Ebenso wie die elektrische Feldstärke und Flussdichte sind auch die beschreibenden Feldgrößen des magnetischen Feldes proportional zueinander und lassen sich über eine Materialkonstante, der sogenannten Permeabilität $\mu$, ineinander umrechnen:

\begin{equation}
    \vec B = \mu \cdot \vec H = \mu_0 \mu_r \cdot \vec H \qquad \quad \text{mit} \qquad \mu_0 = 4\cdot \pi \cdot 10^{-7} \; \si{\volt\second\per\ampere\per\meter}.
\end{equation}

Die relative Permeabilität $\mu_r$ ist ein Materialparameter und die Permeabilität des Vakuums $\mu_0$, ebenso wie $\varepsilon_0$, eine Naturkonstante \cite{EM_Schirmung}.
\par
\vspace{\linespace}
Eine Verknüpfung des elektrischen Feldes mit dem magnetischen kann über den ein einem Leiter hervorgerufenen Stromfluss erfolgen. Die verknüpfende Größe der erzeugten Stromdichte $\vec j$,

\begin{equation}
    \vec j = \sigma \cdot \vec E,
\end{equation}

ist dabei die Leitfähigkeit $\sigma$ des betrachteten Leitermaterials.


\subsection{Verhalten elektrischer und magnetischer Felder an Grenzflächen}

In den vorangegangenen \Abschnitten \ref{cha:2_sub_Elektrisches_Feld} und \ref{cha:2_sub_Magnetisches_Feld} wurde die Abhängigkeit der Flussdichte- und Feldstärkevektoren voneinandern mithilfe von Materialeigenschaften beschrieben. Breiten sich die Felder also entlang verschiedener Materialien mit unterschiedlichen Dielektrizitäten und Permeabilitäten aus, ändern sich beim Übergang zwangsläufig die Ausbreitungsbedingungen. Dies soll im Folgendes betrachtet werden, da das Verhalten der Felder an Materialgrenzflächen grundlegend für die Untersuchtung von Materialschirmen ist. 
\par
\vspace{\linespace}
Für die folgende Betrachtung sei angenommen, dass die untersuchte Grenzfläche relativ zum felderfüllten Raum eine allgemeine Lage aufweist und somit die jeweiligen Feldvektoren schräg auf die Grenzfläche auftreten (vgl. \Abb )  




