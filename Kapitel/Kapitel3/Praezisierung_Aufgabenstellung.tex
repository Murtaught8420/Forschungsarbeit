


Im Verlauf der Aufgabenanalyse erfolgt die Erstellung einer Anforderungsliste, welche die Forderungen an die Konstruktion möglichst umfangreich zusammenstellt. Die Anforderungen setzen sich hauptsächlich aus den Rahmenbedingungen und Wünschen hinsichtlich verschiedener Hauptmerkmale und zur Funktionserfüllung notwendigen Forderungen zusammen.
\par
\vspace{\linespace}
Angelehnt an die Kategorisierung, wie sie in der Literatur beschrieben wird, wurde die Unterteilung in \acp{F} und \acp{W} vorgenommen. Dadurch konnten Anforderungen identifiziert werden, die durch alle Konzepte unbedingt zu erfüllen sind und solche, die nach Möglichkeit Berücksichtigung finden sollten. Der Erfüllungsgrad aller Anforderungen bildete die Bewertungsgrundlage der Konzeptideen.
\par
\vspace{\linespace}
Die Präzisierung der Detailanforderungen erfolgte teilweise und insofern möglich in Iterationsschleifen während der Konzept- und Entwurfsphase, wobei hier nur das Ergebnis dargestellt werden soll. Die Anforderungsliste im \Anhang\ref{A:Anforderungsliste} enthält bei Forderungen, die im Verlauf des Entwicklungsprozesses konkretisiert wurden, zusätzlich und falls notwendig eine Bemerkung zur Hauptquelle für die Präzisierung.





