

\chapter{Zusammenfassung und Ausblick}\label{cha:5}

Das Ziel der vorliegenden Arbeit bestand in der Auslegung einer geeigneten Testumgebung zur Vermessung der Schirmdämpfung von Folien und Schäumen im Fernfeld einer Hornantenne. Zur Charakterisierung der am ILR der TUD entwickelten EMV Schutzfolien auf Basis von CNT sollte im weiteren Verlauf anhand der ermittelten Designkriterien ein Teststand entworfen und aufgebaut werden.
\par
\vspace{\linespace}
Als Ausgangspunkt für weiterführende Betrachtungen wurde dafür zunächst eine Literaturrecherche zu den Grundlagen elektromagnetischer Verträglichkeit und Schirmung sowie den Eigenschaften elektromagnetischer Wellenfelder und deren Ausbreitung durchgeführt. Weiterhin wurden Methoden der Schirmdämpfungsmessung nach aktuellem Stand der Technik betrachtet und deren Eignung für diese Arbeit untersucht.
\par
\vspace{\linespace}
Nach der Auswahl der Messmethode und den Abmessungen der Versuchsumgebung entsprechend der Fernfelddefinition unterschiedlicher Veröffentlichungen erfolgte vorbereitend auf die Konstruktion des Versuchsstandes die Ausarbeitung einer Anforderungsliste. Entsprechend der Funktionsstruktur wurden Wirkprinzipien als Basis für den konstruktiven Entwurf ausgearbeitet. Mithilfe der vorgestellten Methodik wurde der Versuchstand konstruktiv umgesetzt und technisch realisiert.
\par
\vspace{\linespace}
Im Anschluss an die Fertigung und den Aufbau sollte der Teststand mithilfe von bereits durchgeführten Vergleichsmessungen validiert werden. Dafür wurde zunächst die eingesetzte Messtechnik charakterisiert. Durch die Untersuchung der Wirksamkeit kleiner Anpassungen am Aufbau konnte außerdem die Signalqualität für die durchgeführten Schirmdämpfungsmessungen gesteigert werden. Abschließend erfolgte die Vermessung verschiedener frequenzselektiver Proben sowie Buckypaper. Die Bandpässe der frequenzselektiven Oberflächen konnten dabei mit sehr geringen Abweichungen zu den Vergleichsmessungen ermittelt werden. Somit ist der fertige Teststand als Ergebnis dieser Arbeit validiert und zur Charakterisierung frequenzselektiver EMV Schutzfolien einsatzbereit.
\par
\vspace{\linespace}
Auf Basis der durchgeführten Untersuchungen sind weitere Versuche bezüglich des Einflusses der Probenhalterung auf die Qualität des Messsignals denkbar. Zur Reduktion der Beeinflussung durch sekundärer Wellenfronten könnten ebenfalls Messungen im Zeitbereich durchgeführt werden. Abgesehen davon bieten auch die Kalibrationsmethoden des VNA mögliche Ansatzpunkte zur weiteren Reduktion auftretender Fluktuationen des Messsignals.
\par
\vspace{\linespace}
Neben den weiterführenden Betrachtungen in Bezug auf den Versuchsstandes sind weitere Untersuchungen hinsichtlich frequenzselektiver Proben möglich. Dies beinhaltet eine Erweiterung der untersuchten Faktorvarianten, aber auch Messreihen auf Grundlage statistischer Versuchsplanung, um den Einfluss zusätzlicher Parameter wie Einfalls- und Polarisationswinkel zu untersuchen. Damit ließe sich das Verständnis frequenzselektiver Oberflächen erweitern und eine verbesserte Grundlage für deren Anwendung in der elektromagnetischen Verträglichkeit schaffen.









%Zeitbereichsmessungen zur Verbesserung der Genauigkeit und Reduktion des Einflusses von Reflektionen

%andere Kalibtraionsmethoden testen, zum Beispiel SSST (besser bei hohen Frequenzen), SOLR (keine gutes Thru)

%Kupferplatte vor Schrauben des PH stecken, um Einfluss der Muttern zu verringern

%andere Antennenform (siehe Notiz zur Vorstellung des VNA)

%Vollständiges Umkleben der Leiterplatten mit Tape zur Reduktion des Einflusses der Probendicke

%Für Auswertungsprogramm und Filterung automatische Anpassung der Filterparameter, sodass Kurve die Charakteristik am besten erfasst



%---> Beispiele für Anpassungen in Manual schreiben