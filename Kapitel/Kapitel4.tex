
\chapter{Validierung des Teststandes}\label{cha:4}


\section{Bestimmung der Schirmdämpfung mittels VNA}\label{cha:4_Allgemeines}



Im \Abschnitt\ref{cha:2_sub_Begriff_der_Schirmdaempfung} wurde bereits auf den Begriff der Schirmdämpfung als Pegelmaß der Feldgrößen\-beträge (vgl. \Gleichungen\eqref{eq:2_Schirmdaempfung_elektrisch} bis \eqref{eq:2_Schirmdaempfung_Leistung}) eingegangen. Bei der Bestimmung der Schirmdämpfung mit einem VNA eignet sich jedoch die Betrachtung der Messstrecke mithilfe der Zweitortheorie. Dabei stellt die Messstrecke einen passiven Zweipol dar, der sich analog zu einem passiven elektrischen Bauelement verhält und durch Wellengrößen eindeutig beschrieben werden kann. In der \Abb\ref{fig:4_Zweitor} ist dies schematisch dargestellt, wobei die jeweils zulaufenden Wellen durch die Größen $\underline{a}_1$ und $\underline{a}_2$ und die ablaufenden mit $\underline{b}_1$ und $\underline{b}_2$ gekennzeichnet sind~\cite{Taschenbuch_HF-Technik}. 
\par
\vspace{\linespace}


\begin{figure}[ht]
    \centering
    \includegraphics[page = 1, trim = 5.9cm 24cm 5.9cm 1.3cm, clip, width = 0.5\textwidth]{Abbildungen/Kapitel4/Zweitor.pdf}
    \caption[Zweitor mit Wellengrößen]{Zweitor mit Wellengrößen nach~\cite{Taschenbuch_HF-Technik}}
    \label{fig:4_Zweitor}
\end{figure}


In der Praxis hat die Darstellung mithilfe der Wellengrößen den Vorteil, dass diese auch bei hohen \mbox{Frequenzen} direkt gemessen werden können~\cite{Taschenbuch_HF-Technik}. Neben der äquivalenten Beschreibung des Verhaltens des Zweitors durch Beziehungen zwischen den Torspannungen und -strömen kann das Übertragungsverhalten auch mithilfe der Streumatrix $\underline{S}$ 


\begin{equation}
\centering
    \left(\begin{array}{c}\underline{b}_1 \\ \underline{b}_2 \end{array}\right) = \left(\begin{array}{cc}\underline{S}_{11} & \underline{S}_{12} \\ \underline{S}_{21} & \underline{S}_{22} \end{array}\right) \left(\begin{array}{c}\underline{a}_{1} \\ \underline{a}_{2} \end{array}\right) \label{cha:4_Streuparameter}
\end{equation}

charakterisiert werden~\cite{Taschenbuch_HF-Technik}. $\underline{S}$ ist für struktursymmetrische Zweitore ebenfalls symmetrisch~\cite{Grundkurs_Hochfrequenztechnik}.  
\par
\vspace{\linespace}
Die Bestimmung der Streuparameter oder S-Parameter $\underline{S}_{11}$ bis $\underline{S}_{22}$ kann nun durch die Beschaltung jeweils eines Tores mit der Systemimpedanz durch den VNA erfolgen. Dies führt im Idealfall zum reflexionsfreien Abschluss des Tores, wodurch die jeweils zulaufende Wellengröße $\underline{a}_1$ bzw. $\underline{a}_2$ verschwindet~\cite{Grundkurs_Hochfrequenztechnik}. Die Berechnung der Streuparameter mithilfe der \Gleichung\eqref{cha:4_Streuparameter} wird damit trivial. 
\par
\vspace{\linespace}
Für die Deutung der S-Parameter ist die Reihenfolge der Indizes relevant. $\underline{S}_{11}$ und $\underline{S}_{22}$ beschreiben den primären bzw. sekundären Reflektionsfaktor bei Abschluss des jeweils anderen Tores. Der Transmissionskoeffizient $\underline{S}_{21}$ charakterisiert die Transmission von 1 nach 2 bei abgeschlossenem Tor 2. Die Bedeutung von $\underline{S}_{12}$ ergibt sich analog durch Vertauschen der Tore. Das Signalflussdiagramm in \Abb\ref{fig:4_Signalflussdiagramm_Zweitor} veranschaulicht die Zuordnung der Indizes zu den einzelnen Streuparametern.
\par
\vspace{\linespace}

\begin{figure}[ht]
    \centering
    \includegraphics[page = 1, trim = 5.9cm 19.2cm 5.9cm 5cm, clip, width = 0.5\textwidth]{Abbildungen/Kapitel4/Zweitor.pdf}
    \caption[Signalflussdiagramm eines Zweitors mit Streuparametern]{Signalflussdiagramm eines Zweitors mit Streuparametern nach~\cite{Grundkurs_Hochfrequenztechnik}}
    \label{fig:4_Signalflussdiagramm_Zweitor}
\end{figure}

Um mit der ermittelten und im Allgemeinen komplexen Streumatrix als Übertragungsmaß zu arbeiten, bietet sich die logarithmische Darstellung an, weshalb die Angabe der S-Parameter oft in Dezibel erfolgt~\cite{Grundkurs_Hochfrequenztechnik} 

\begin{equation}
    S_{12, dB} = 20 \cdot \lg \left( \left| \underline{S}_{12} \right| \right) \; \text{.} \label{eq:4_Schirmdaempfung_Dezibel}
\end{equation}


Mithilfe der Transmissionskoeffizienten kann die Schirmdämpfung eines Materials durch Einfügungs\-messung ermittelt werden (vgl. \Abschnitte\ref{cha:2_sub_Begriff_der_Schirmdaempfung} und \ref{cha:2_Methoden_der_Schirmdaempfungsmessung}). In der \Abb\ref{fig:4_Einfuegungsmessung} ist die Vorgehensweise schematisch dargestellt. Beim Arbeiten mit den S-Parametern in Dezibel kann die Schirmdämpfung durch Subtraktion der ermittelten Freiraumdämpfung ohne Schirm~\mbox{(\subref{subfig:4_Einfuegungsmessung_Fall1})} von den Messwerten mit Schirm~\mbox{(\subref{subfig:4_Einfuegungsmessung_Fall2})} bestimmt werden. Dies wird für alle Frequenzen des Messbereichs durchgeführt.
\par



\begin{figure}[ht]
    \centering
    \begin{subfigure}[b]{0.99\textwidth}
        \includegraphics[page = 1, width=\textwidth, trim = 0cm 14.75cm 0cm 2.5cm, clip]{Abbildungen/Kapitel4/Einfuegungsmessung.pdf}
        \caption{\label{subfig:4_Einfuegungsmessung_Fall1}}
    \end{subfigure}
    \vspace{0.5cm}
    \begin{subfigure}[b]{0.99\textwidth}
       \includegraphics[page = 1, width=\textwidth, trim = 0cm 17.5cm 0cm 0cm, clip]{Abbildungen/Kapitel4/Einfuegungsmessung.pdf}
        \caption{\label{subfig:4_Einfuegungsmessung_Fall2}}
    \end{subfigure}
    \caption[Schematische Darstellung der durchgeführten Einfügungsmessung]{Schematische Darstellung der durchgeführten Einfügungsmessung im Fernfeld ohne~(\subref{subfig:4_Einfuegungsmessung_Fall1}) und mit~(\subref{subfig:4_Einfuegungsmessung_Fall2}) Schirm im Koppelpfad}
    \label{fig:4_Einfuegungsmessung}
\end{figure}


Da es sich bei den verwendeten Schirmen um rein passive Elemente handelt und im Idealfall das Reziprozitätsgesetz gilt (vgl. \Abschnitt\ref{cha:2_sub_Begriff_der_Schirmdaempfung}), sollten die Transmissionskoeffizienten der Streumatrix gleich sein. Da es trotz sorgfältiger Abdeckung der meisten reflektiven Flächen innerhalb der Testkammer Reflektionen und damit zusätzlichen Wellenfronten gibt, gilt 

\begin{equation}
    |S_{12}| \approx |S_{21}|\label{eq:4_Naeherung_Transmissionskoeffizienten}
\end{equation}

nur noch in Näherung~\cite{EM_Schirmung}. Im \Abschnitt\ref{cha:4_Versuchsvorbereitung_und_Durchfuehrung} wird darauf anhand der Messwerte noch einmal eingegangen.
\par
\vspace{\linespace}




%Bei Messung der Schirmdämpfung --> da rein passiv und nach Reziprozitätsgesetz sollte Streumatrix ebenfalls symmetrisch sein --> kann an Messwerten angelesen werden (ggf. Vergleich einfügen)

%Einfügungsmessung erklären mit Freiraumdämpfung, etc.


Für die durchgeführten Messungen wird ein 2-Port VNA genutzt. Damit ist die Bestimmung aller Streuparameter möglich. Der VNA gibt dabei das Testsignal aus und übernimmt gleichzeitig die Analyse und Bestimmung der Amplitude und Phase des empfangenen Signals. Auch wenn zur reinen Bestimmung der Schirmdämpfung nach \Gleichung\eqref{eq:4_Schirmdaempfung_Dezibel} das komplexe Ausgangsignal wieder auf einen skalaren Wert reduziert wird, kann mithilfe der Information über die Phasenlage des Signals eine höhere Genauigkeit erzielt werden~\cite{VNA-Handbuch}. Außerdem ermöglicht es zusätzliche Messmethoden wie Zeitbereichsmessungen und die Darstellung von Smith-Diagrammen. Letztere werden vor allem für die Auswertung der Reflektionsparameter genutzt.

%2-Port VNA --> Quelle und Analyzer gleichzeitig

%Modellnummer und Messbereich





\section{Charakterisierung und Kalibration der Messtechnik}\label{cha:4_Kalibration_Messtechnik}



Das Messsignal des VNA setzt sich aus Nutz- und Störsignal zusammen. Letzteres entsteht bspw. durch äußere Einflüsse oder die Temperaturabhängigkeit bestimmter Bauteile. Um eine Abschätzung der Messunsicherheit zu treffen, sollen die Störsignale kurz untersucht werden.
\par
\vspace{\linespace}
Durch eine Messung mit ungeschirmter Messtrecke (vgl. \Abb\ref{subfig:4_Einfuegungsmessung_Fall1}) kann das zufällige Rauschsignal erfasst werden, welches in \Abb\ref{fig:4_Stoersignal} dargestellt ist. Erkennbar ist, dass die Charakteristik tatsächlich der eines zufälligen Rauschens entspricht mit einem maximalen Amplitudenbetrag von etwas weniger als \SI{0.075}{\Dezibel} und damit deutlich unter den erwarteten Schirmdämpfungswerten der Proben. Bezogen auf die Feldgrößen entspricht dies nach \Gleichung\eqref{eq:2_Relativer_Pegel_Feldgroessen} einer Messunsicherheit von etwa \SI{0.87}{\percent}. Gegenüber anderen Einflüssen wie sekundären Wellenfronten aufgrund von Reflektionen oder Interferenzen durch Bauteile im Bereich der ersten Fresnelzone ist dies, vor allem im unteren Frequenzbereich, zu vernachlässigen. 
\par


\begin{figure}[ht]
    \centering
    \includegraphics[page = 1, width = .99\textwidth, trim = 0cm 18.05cm 0.3cm 0.05cm, clip]{Abbildungen/Kapitel4/Rauschsignal.pdf}
    \caption{Zufälliges Störsignal des VNA von \SI{1}{\giga\hertz} bis \SI{18}{\giga\hertz}}
    \label{fig:4_Stoersignal}
\end{figure}


Das Rauschsignal weist weiterhin eine deutliche Erhöhung der Amplitude ab ca. \SI{8}{\giga\hertz} auf. Da diese Art von statistischem Rauschen nicht von den Fehlerkorrekturmethoden der Kalibration reduziert wird, könnte dies auf den internen Aufbau des VNA oder auf eine höhere Rauschanfälligkeit der Komponenten zurückzuführen sein. Auf die korrigierbaren Fehlerterme soll im Folgenden kurz eingegangen werden.
% entweder auf den internen Aufbau des VNA oder die verwendete Kalibrationsmethode zurückzuführen sein. Im Folgenden soll deshalb außerdem auf die Kalibration des VNA etwas näher eingegangen werden.
\par
\vspace{\linespace}
In Abhängigkeit der Anzahl der verwendeten Ports des VNA können unterschiedlich viele Korrekturterme auf die Messung angewandt werden. Hier werden stets beide Ports zur Messung benötigt, sodass nur auf das gängigste Korrekturmodell eingegangen werden soll. Da der VNA den Phasor des Messsignals aufzeichnet, können je Port sechs Fehlerterme korrigiert werden~\cite{VNA-Handbuch}. In der \Abb\ref{fig:4_Fehlermodell} ist das vom VNA verwendete Fehlermodell mit den Korrekturtermen $e_{xx}$ für eine vollständige 2-Port Kalibration in Vorwärtsrichtung dargestellt. Wird diese in beide Richtungen angewandt, erhält man die geläufige \mbox{12-Term} Kalibration nach dem aktuellen Stand der Technik~\cite{VNA_Error_Models_and_Calibration_Methods}. Auf die zugrunde liegende Theorie soll an dieser Stelle nicht näher eingegangen werden. Alternativ ist die 1-Pfad-2-Port-Kalibration, welche in der Software des VNA nutzbar ist, ebenfalls ausreichend, wobei bspw. nur die Vorwärts-Koeffizienten der Streumatrix korrigiert werden. 
\par
\vspace{\linespace}


\begin{figure}[ht]
    \centering
    \includegraphics[page = 1, trim = 2cm 12cm 2cm 10.3cm, clip, width = 0.9\textwidth]{Abbildungen/Kapitel4/Zweitor.pdf}
    \caption[Signalflussdiagramm des 12-Term Fehlermodells in Vorwärtsrichtung für zwei Ports]{Signalflussdiagramm des 12-Term Fehlermodells in Vorwärtsrichtung für zwei Ports nach~\cite{VNA_Error_Models_and_Calibration_Methods}}
    \label{fig:4_Fehlermodell}
\end{figure}

Die empfohlene Kalibrationsmethode mithilfe des vorhandenen Kalibrationskits, das entsprechend hochpräzise definierte Standards enthält, ist die sogenannte SOLT-Kalibration (Short-Open-Load-Through). Dabei werden die Ports des VNA jeweils nacheinander vollständig abgeschlossen, geöffnet und mit einer Last von \SI{50}{\ohm} beaufschlagt. Anschließend werden beide Ports über das Kit miteinander verbunden. Vorteilhaft bei dieser Methode ist, dass sie nicht bandbegrenzt und relativ einfach durchzuführen ist. Die Genauigkeit hängt hier vor allem von der Definition der Standards ab. Andere Methoden sind jedoch vor allem für sehr hohe Frequenzen etwas genauer~\cite{VNA-Calibration_Application_Note}. 
\par
\vspace{\linespace}
Durchgeführt werden kann die Kalibration natürlich nur an der Schnittstelle der Signalkabel mit den Antennen, sodass die Einflüsse der Antennen nicht korrigiert werden können. Außerdem ist anzumerken, dass zur Durchführung der \glqq Through\grqq-Kalibration nach dieser Methode eine große Lageveränderung der Signalkabel notwendig ist, da diese mithilfe des Kalibrationskits verbunden werden müssen. Dies kann durchaus einen nicht zu vernachlässigenden Einfluss auf das Messsignal haben, sodass ggf. die in~\cite{VNA-Calibration_Application_Note} und~\cite{VNA-Handbuch} beschriebene SOLR-Methode (Short-Open-Load-Reciprocal), welche keine direkte \glqq Through\grqq-Definition benötigt, noch bessere Ergebnisse liefert. Dies wurde im Rahmen dieser Arbeit nicht getestet, könnte aber für fortführende Untersuchungen in Betracht gezogen werden.
\par
\vspace{\linespace}
Zu beachten ist außerdem, dass die Genauigkeit nach erfolgter Kalibration aufgrund von bspw. Drifts abnehmen kann, sodass in Abhängigkeit der Nutzung und Umgebungsbedingungen eine Aktualisierung der gespeicherten Kalibrierung notwendig ist. Da im Labor relativ konstante Umgebungsbedingungen herrschen und die Kalibration nach einer entsprechenden Aufwärmzeit des VNA durchgeführt wurde, verlängert sich die Zeit zwischen den notwendigen Neukalibration~\cite{VNA-Handbuch, VNA_Error_Models_and_Calibration_Methods}. Es wird jedoch empfohlen, diese vor jeder Messkampagne durchzuführen.


%Wie alle Messgeräte --> zufällige SChwankungen
%Charakterisierung des Rauschsignals und kurzer Vergleich mit Größ0enordnung der Messungen


%--> Kalibration zum Ausgleich von Drifts und Offsets gegenüber einem idealen Messgerät

%verschiedene Methoden --> beschreiben

%auf SLOT und Thru eingehen und vergleichen, wann welche Methode sinnvoll ist


%Neukalibration nach Umgebungsbedingungen --> Temperatur und gemetrischen Änderungen; Thru-Update möglich bei erfolgter 2-Port oder 1 Path 2Port Kalibration

%Interpolation beachten bei erfolgter Kalibration --> Wenn aus müssen Messpunkte mit Kalibrationspunkten übereinstimmen, ansonsten wird interpoliert --> aufpassen, dass Signal richtig abgebilet wird!



%langes Einschalten vor Kalibrierung --> Drifts durch Temperatur etc. entgegengewirkt (SCHULZE, K.: Experimentelle Messtechnik im Maschinen- und Stahlbau. 1. Auflage. Berlin: VEB Verlag Technik, 1988)

%Kalibrierung beschreiben
    %Darauf achten, dass Anzahl der Messpunkte und Bereich VOR Kalibration eingestellt wird
    %Beschreiben, worauf bei Kalibration noch zu achten ist
    
    
    
    
    
    
\section{Versuchsvorbereitung und -durchführung}\label{cha:4_Versuchsvorbereitung_und_Durchfuehrung}



\subsection{Versuchsbedingungen}

Um im Rahmen der Versuchsauswertung im \Abschnitt\ref{cha:4_Auswertung} eine Verifikation des Teststandes vornehmen zu können, wurden vor allem Messungen mit Leiterplatten durchgeführt, die mit entsprechenden frequenzselektiven Oberflächen gefertigt wurden. Für diese Proben existieren Vergleichsmessungen hoher Qualität vom Institut für Nachrichtentechnik der TU Dresden~\cite{FSS_Toedter_Diplomarbeit}. Da die Änderung der elektrischen Leitfähigkeit des Kupfers der Leiterplatten, welche für die Schirmdämpfung nicht magnetischer Materialien bei gegebener Frequenz vor allem ausschlaggebend ist (vgl. \Abschnitt\ref{cha:2_sub_Schirmung_ebener_Wellenfelder}), für Schwankungen der Umgebungsbedingungen in den Laborräumen vernachlässigbar ist~\cite{Materialdaten_Kupfer}, ist vor allem die Sensitivität des VNA gegenüber wechselnden Betriebstemperaturen zu beachten.
\par
\vspace{\linespace}
Um vergleichbare Messungen über der gesamten Messdauer zu erhalten wurden diese zur Verringerung von Temperaturdrifts erst nach einer ausreichenden Aufwärmzeit des VNA durchgeführt (vgl. \Abschnitt\ref{cha:4_Kalibration_Messtechnik}). Die Betriebsbedingungen in den Laborräumen entsprechend des Weiteren weitestgehend denen der Erstkalibration und Einrichtung des VNA beim Hersteller laut des entsprechenden Protokolls. Es kann somit davon ausgegangen werden, dass entstehende Abweichungen zwischen Messungen durch Temperaturschwankungen minimal sind. Für die verwendeten Pyramidenabsorber gilt außerdem die angegebene garantierte Reflektionsdämpfung in einem weiten Temperaturbereich, sodass auch hier keine großen Schwankungen zu erwarten sind~\cite{Eco_Messtechnik_Absorber}.
%Tempraturabhängigkeit Absorber

%Versuchsbedingungen --> siehe Prjektarbeit


\subsection{Versuchsvorbereitung}

Die abschließenden Vorbereitungen vor der Durchführung der Messungen beinhalten neben dem Anschluss aller Signalkabel und des VNA auch das Aufstellen der Antennen und die Platzierung des Reflektors so, dass eine direkte Sichtverbindung beider Antennen nur durch die Probenöffnung möglich ist. Da es sich bei den Antennen um Richtantennen handelt, ist die Ausrichtung zueinander unter Beachtung der Polarisationsrichtung von besonderer Bedeutung. Dies wurde durch sorgfältige Vermessung sichergestellt, ebenso wie die vertikale Ausrichtung des Reflektors entsprechend der gestellten Anforderungen. Die aufgebaute Messstrecke ist in der \Abb\ref{fig:4_Messstrecke} zu sehen.
\par
\vspace{\linespace}

\begin{figure}[ht]
    \centering
    \includegraphics[height=.2\textheight, draft = true]{Abbildungen/Kapitel4/IMG_5660.jpg}
    \hspace{1cm}
    \includegraphics[height=.2\textheight, draft = true]{Abbildungen/Kapitel4/IMG_5665.jpg}
    \caption{Messstrecke ohne und mit Reflektor}
    \label{fig:4_Messstrecke}
\end{figure}


%Ausrichtung der Antennen genau
%Ebene der Probekörper genau


Ein weiterer wichtiger Schritt der Versuchsvorbereitung ist die korrekte Verbindung aller Signalkabel mit den entsprechenden Konnektoren, sowohl zur Kalibration als auch für die eigentlichen Messungen. Durch eine unsauber Verschraubung an der Schnittstelle zweier Kabel steigt der Wellenwiderstand, sodass aufgrund des Impedanzsprunges eine Grenzfläche entsteht. Dies führt entsprechend den Betrachtungen in \Abschnitt\ref{cha:2_sub_Verhalten_an_Grenzflächen} zu ungewollten Reflektionen innerhalb der Signalleitung und kann das Messsignal störend beeinflussen. Häufig ist eine schlechte Verbindung auch an einem instabilen Signal zu erkennen. Bei der Verschraubung ist daher stest auf die korrekte Ausrichtung der Konnektoren zueinander zu achten und die Verbindung ist nach vorsichtigem Anziehen per Hand mittels eines Drehmomentenschlüssels unter Verwendung des korrekten Drehmoments zu sichern.
\par
\vspace{\linespace}
Abschließend wurde die gesamte Schirmhülle an der dafür vorgesehenen Adapterplatte, an der sich auch die Konnektoren der Signalkabel befinden, mit Erdpotenzial verbunden. Dadurch können entstehende Störströme abgeleitet werden, sodass die Schirmungseffektivität dadurch nicht beeinträchtigt wird. 
\par
\vspace{\linespace}
Im Folgenden soll der allgemeine Ablauf einer vollständigen Messung, der für alle untersuchten Probetypen gleich ist, anhang der durchzuführenden Schritte dargestellt werden. 


\subsection{Versuchsdurchführung}

Zu Beginn jeder Messreihe wird eine kurze Sichtinspektion der Testkammer durchgeführt, um fehlerhafte Messergebnisse aufgrund von Beschädigungen oder Defekten schon zu Beginn ausschließen zu können. Im Anschluss erfolgt die Vorbereitung der Messung in der Shockline Software des VNA. Dies schließt die Kalibration der Messtechnik entsprechend den Ausführungen in \Abschnitt\ref{cha:4_Kalibration_Messtechnik} ein.
\par
\vspace{\linespace}
Nach Abschluss der Vorbereitungen kann die Messung der sogenannten Freiraumdämpfung erfolgen. Das bedeutet dass die Schirmdämpfung der Messstrecke einschließlich des Probenhalters, jedoch ohne Probe, ermittelt wird. Dabei ist sicherzustellen, dass die beiden Teile des Probenhalters trotzdem fest miteinander verschraubt sind. Ist dies erfolgt kann die Probe eingebracht werden. Auch hier ist zur Herstellung eines leitfähigen Kontaktes um den gesamten Rand des Probekörpers mit dem Probenhalter auf eine feste Verschraubung zu achten. Um dies zu gewährleisten wurden die vermessenen Leiterplatte bspw. mit Aluminiumband umklebt. Die eingebaute Probe ist in \Abb\ref{fig:4_Probenhalter_mit_Probe} dargestellt. Dort ist ebenfalls die Struktur zu erkennen, welche für eine frequenzselektive Dämpfung der Transmission sorgt. Diese besteht bei den vorliegenden Probekörpern im Wesentlichen aus kreuzförmigen Mustern unterschiedlicher Balkenbreite und -länge, welche die leitfähige Schicht der Leiterplatten unterbrechen. Für eine detaillierte Betrachtung der Wirkungsweise solcher frequenzselektiven Oberflächen sei auf die Arbeit in~\cite{FSS_Toedter_Diplomarbeit} verwiesen.
\par
\vspace{\linespace}


\begin{figure}[ht]
    \centering
    \includegraphics[height=.2\textheight, draft = true]{Kapitel/Kapitel4/Probenhalter.jpg}
    \hspace{1cm}
    \includegraphics[height=.2\textheight, draft = true]{Abbildungen/Kapitel4/IMG_5675_trimmed.jpg}
    \caption{Probenhalter mit eingebauter Probe und Messstrecke mit Probenhalter}
    \label{fig:4_Probenhalter_mit_Probe}
\end{figure}


Die Ermittlung der S-Parameter mit Probe erfolgt analog zur Messung der Freiraumdämpfung. Die Einfüge- oder Schirmdämpfung kann dann durch Subtraktion der Dämpfung der Messstrecke von der ermittelten Gesamtdämpfung mit Probe berechnet werden (vgl. \Abschnitt\ref{cha:4_Allgemeines}). Zu beachten ist dabei in welcher Reihenfolge die Ports des VNA an die Antennen angeschlossen wurden, sodass aus den gespeicherten Messwerten der korrekte Transmissisonskoeffizient zur Berechnung der Schirmdämpfung genutzt werden kann.
\par
\vspace{\linespace}
Im folgenden \Abschnitt werden die Ergebnisse der durchgeführten Messungen vorgestellt und ausgewertet.




%Verschraubung Probenhalter

%Messung ohne Probe

%Verschraubung Probe

%Messung mit Probe

%Differenz beider Messungen = S21











\section{Versuchsauswertung}\label{cha:4_Auswertung}











%kein Unterschied feststellbar bei zwei Messungen ohne Probe an unterschiedlichen Tagen, da etwas gleiche Bedingungen und keine Veränderungen an Signalkabeln, etc zwischen Messungen nach Kalibration


%Ohne Erdung verstärkt sich Rauschen leicht, vor allem im hohen Frequenzbereich; kaum Unterschieschied, ob nur mit Schirm verbunden oder auch geerdet

%Absorber vor Antennen hat vor allem Auswirkungen im unteren Frequenzbereich
%gleiches gilt für Absorber unter Reflektor zur Reduktion von sekundären Wellenfronten --> Aussage aus \cite bestätigt, kein Einfluss ab ..GHz feststellbar

%Abkleben der Schnittkante des Reflektors hat geringen Einfluss, vor allem im hohen Frequenzbereich

%Peak aufgrund von Cut-Off und Beugungserscheinungen --> je besser Rest der Kurve aussieht, desto steiler wird Peak --> letztendlich bei 1,12965 GHz = 26,5385 cm Wellenlänge



%--> Schirmdämpfungsplots mit verschiedenen Anpassungen in ein Diagramm
        %-erste Messung, keine Anpassung
        %-Erdung
        %-normale Schrauben
        %-Absorber an Reflektor (hinten)
        %-Absorber an Reflektor (beidseitig)
        %-normale Schrauben und Abstand bei Messung ohne Probe
        %-keine Schrauben
        %-Absorber an PH



%An allen Varianten kann wie vermutet festgestellt werden, dass Reziprozität gilt, d.h. Werte für S12 und S21 sind bis auf das statistische Rauschen gleicht




% \begin{figure}[ht]
%     \centering
%     \includegraphics[page = 1, width = 0.99\textwidth, trim = 0cm 14.3cm 0cm 0cm, clip]{Abbildungen/Kapitel4/Messergebnisse/10k0x10k0-1.pdf}
%     \caption{Caption}
%     \label{fig:my_label3}
% \end{figure}