\chapter*{Abkürzungsverzeichnis}
\addcontentsline{toc}{chapter}{Abkürzungsverzeichnis}
%\thispagestyle{fancy}
\markboth{Abkürzungsverzeichnis}{Abkürzungsverzeichnis} 

\noindent\rule{\textwidth}{0.5pt}
\textbf{Abkürzung} \hspace{18mm} \textbf{Erläuterung} \\[-\linespace]
\noindent\rule{\textwidth}{0.5pt}

\begin{acronym}[langesPlatzhalterwort]
%\acro{<Kurzform>}{<Langform>}
\acro{CE}{Conformité Européenne (Europäische Konformität)}
\acro{CNT}{Kohlenstoffnanoröhrchen (Carbon Nanotubes)}
\acro{EMV}{Elektromagnetische Verträglichkeit}
    \acrodefplural{EMV}[EMV]{Elektromagnetischen Verträglichkeit}
\acro{F}{Forderung}
    \acrodefplural{F}[F]{Forderungen}
\acro{HF}{Hochfrequenz}
\acro{K}{Kriterium}
    \acrodefplural{K}[K]{Kriterien}
\acro{MSC}{Modenverwirbelungskammer (Mode Stirred Chamber)}
\acro{NE-Metalle}{Nichteisen-Metalle}
    \acrodefplural{NE-Metalle}{Nichteisen-Metallen}
\acro{PU}{Polyurethan}
\acro{SMA}{Sub-Miniature-A}
\acro{TEM}{Transversalelektromagnetische Wellen (Transverse Electromagnetic Mode)}
\acro{TUD}{Technische Universität Dresden}
\acro{VNA}[Netzwerkanalysator]{Vektorieller Netzwerkanalysator}
\acro{VSWR}[Stehwellenverhältnis]{Stehwellenverhältnis (Voltage Standing Wave Ratio)}
\acro{W}{Wunsch}
    \acrodefplural{W}[W]{Wünsche}

\end{acronym}