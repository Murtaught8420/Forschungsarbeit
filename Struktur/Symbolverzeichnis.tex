\chapter*{Symbolverzeichnis}
\addcontentsline{toc}{chapter}{Symbolverzeichnis}
%\thispagestyle{fancy}
\markboth{Symbolverzeichnis}{Symbolverzeichnis}


\subsection*{Variablen}
\vspace{\linespace}
\textit{Lateinische Symbole} \\[.5\linespace]
\noindent\rule{\textwidth}{0.5pt}
\textbf{Symbol} \hspace{12.5mm} \textbf{Einheit} \hspace{10.5mm} \textbf{Bezeichnung} \\[-\linespace]
\noindent\rule{\textwidth}{0.5pt}

\begin{acronym}[Platzhalterwort]

\acro{a}[$a$]{\acrounit{\meter}Geometrische Abmessung}
\acro{a_e}[$a_e$]{\acrounit{\Dezibel}Elektrische Schirmdämpfung}
\acro{a_m}[$a_m$]{\acrounit{\Dezibel}Magnetische Schirmdämpfung}
\acro{a_S}[$a_S$]{\acrounit{\Dezibel}Schirmdämpfung}
\acro{vec_a}[$a(\vec x,\ldots)$]{\acrounit{-}Allgemeines Skalarfeld}
\acro{A_P}[$A_P$]{\acrounit{\square\meter}Kondensatorplattenfläche}
\acro{A_w}[$A_w$]{\acrounit{\Dezibel}Absorbtionsverlust ebener Wellen}
\acro{vec_A}[$\vec A(\vec x,\ldots)$]{\acrounit{-}Allgemeines Vektorfeld}

\acro{b}[$b$]{\acrounit{\meter}Geometrische Abmessung}
\acro{B_w}[$B_w$]{\acrounit{\Dezibel}Korrekturfaktor für Sekundärreflektionen im Schirmmaterial}
\acro{vec_B}[$\vec B(\vec x, \ldots)$]{\acrounit{\tesla}Magnetische Flussdichte}
\acro{vec_B_1}[$\vec B_{1\ldots2}$]{\acrounit{\tesla}Magnetische Flussdichte im Material 1 / 2}
\acro{vec_B_n}[$\vec B_n$]{\acrounit{\tesla}Normalkomponente der magnetischen Flussdichte}
\acro{vec_B_t}[$\vec B_t$]{\acrounit{\tesla}Tangentialkomponente der magnetischen Flussdichte}

\acro{c}[$c$]{\acrounit{\meter}Geometrische Abmessung}

\acro{d}[$d$]{\acrounit{\meter}Durchmesser}
\acro{d1}[$d_1$]{\acrounit{\meter}Abstand Sendeantenne Probekörper}
\acro{d2}[$d_2$]{\acrounit{\meter}Abstand Empfangsantenne Probekörper}
\acro{d_S}[$d_S$]{\acrounit{\meter}Schirmabmessung}
\acro{D}[$D$]{\acrounit{\meter}Dipollänge / Größte Antennenausdehnung}
\acro{D_II}[$D_{\uproman{2}}$]{\acrounit{1}Durchlassfaktor Betrachtungsfall \uproman{2}}
\acro{vec_D}[$\vec D(\vec x, \ldots)$]{\acrounit{\ampere\second\per\square\meter}Elektrische Flussdichte}
\acro{vec_D_1}[$\vec D_{1\ldots2}$]{\acrounit{\ampere\second\per\square\meter}Elektrische Flussdichte im Material 1 / 2}
\acro{vec_D_n}[$\vec D_n$]{\acrounit{\ampere\second\per\square\meter}Normalkomponente der elektrischen Flussdichte}
\acro{vec_D_t}[$\vec D_t$]{\acrounit{\ampere\second\per\square\meter}Tangentialkomponente der elektrischen Flussdichte}


\acro{vec_e_x}[$\vec e_x$]{\acrounit{1}Normaleneinheitsvektor in x-Richtung}
\acro{vec_e_y}[$\vec e_y$]{\acrounit{1}Normaleneinheitsvektor in y-Richtung}
\acro{vec_e_z}[$\vec e_z$]{\acrounit{1}Normaleneinheitsvektor in z-Richtung}
\acro{E_0}[$E_0$]{\acrounit{\volt\per\meter}Elektrische Feldstärke am Feldursprung bzw. ungeschirmt}
\acro{E_1}[$E_1$]{\acrounit{\volt\per\meter}Elektrische Feldstärke nach Schirmeinfügung}
\acro{E_r}[$E_r$]{\acrounit{\volt\per\meter}Radialkomponente der elektrischen Feldstärke in Kugelkoordinaten}
\acro{E_yz}[$E_{y\ldots z}$]{\acrounit{\volt\per\meter}Komponente der elektrischen Feldstärke in y- / z-Richtung}
\acro{E_theta}[$E_{\theta}$]{\acrounit{\volt\per\meter}Winkelkomponente der elektrischen Feldstärke in Kugelkoordinaten}
\acro{E_phi}[$E_{\phi}$]{\acrounit{\volt\per\meter}Winkelkomponente der elektrischen Feldstärke in Kugelkoordinaten}
\acro{vec_E}[$\vec E(\vec x, \ldots)$]{\acrounit{\volt\per\meter}Elektrische Feldstärke}
\acro{vec_E_d}[$\vec E_d$]{\acrounit{\volt\per\meter}Vektor der elektrischen Feldstärke nach Grenzflächenübergang}
\acro{vec_E_e}[$\vec E_e$]{\acrounit{\volt\per\meter}Einfallender Vektor der elektrischen Feldstärke}
\acro{vec_E_n}[$\vec E_n$]{\acrounit{\volt\per\meter}Normalkomponente der elektrischen Feldstärke}
\acro{vec_E_r}[$\vec E_r$]{\acrounit{\volt\per\meter}Reflektierter Vektor der elektrischen Feldstärke}
\acro{vec_E_t}[$\vec E_t$]{\acrounit{\volt\per\meter}Tangentialkomponente der elektrischen Feldstärke}


\acro{f}[$f$]{\acrounit{\hertz}Frequenz}
\acro{f_k}[$f_k$]{\acrounit{\hertz}Kritische Frequenz}
\acro{f_R}[$f_R$]{\acrounit{\hertz}Resonanzfrequenz}
\acro{f(xt)}[$f(x,t)$]{\acrounit{-}Allgemeine Funktion}
\acro{F_L}[$F_L$]{\acrounit{\newton}Kraft auf einen stromdurchflossenen Leiter}
\acro{vec_F_q}[$\vec F_q$]{\acrounit{\newton}Kraft auf Ladung im elektrischen Feld}

\acro{g(xt)}[$g(x,t)$]{\acrounit{-}Allgemeine Funktion}
\acro{G_V}[$G_V$]{\acrounit{-}Gesamtwert der Variantenbewertung}

\acro{H_0}[$H_0$]{\acrounit{\ampere\per\meter}Magnetische Feldstärke am Feldursprung bzw. ungeschirmt}
\acro{H_r}[$H_r$]{\acrounit{\ampere\per\meter}Radialkomponente der magnetischen Feldstärke in Kugelkoordinaten}
\acro{H_z}[$H_z$]{\acrounit{\ampere\per\meter}Komponente der magnetischen Feldstärke in z-Richtung}
\acro{H_theta}[$H_{\theta}$]{\acrounit{\ampere\per\meter}Winkelkomponente der magnetischen Feldstärke in Kugelkoordinaten}
\acro{H_phi}[$H_{\phi}$]{\acrounit{\ampere\per\meter}Winkelkomponente der magnetischen Feldstärke in Kugelkoordinaten}
\acro{vec_H}[$\vec H(\vec x, \ldots)$]{\acrounit{\ampere\per\meter}Magnetische Feldstärke}
\acro{vec_H_d}[$\vec H_d$]{\acrounit{\ampere\per\meter}Vektor der magnetischen Feldstärke nach Grenzflächenübergang}
\acro{vec_H_e}[$\vec H_e$]{\acrounit{\ampere\per\meter}Einfallender Vektor der magnetischen Feldstärke}
\acro{vec_H_n}[$\vec H_n$]{\acrounit{\ampere\per\meter}Normalkomponente der magnetischen Feldstärke}
\acro{vec_H_r}[$\vec H_r$]{\acrounit{\ampere\per\meter}Reflektierter Vektor der magnetischen Feldstärke}
\acro{vec_H_t}[$\vec H_t$]{\acrounit{\ampere\per\meter}Tangentialkomponente der magnetischen Feldstärke}


\acro{^i}[$\hat i$]{\acrounit{\ampere}Amplitude der Stromstärke}
\acro{I}[$I$]{\acrounit{\ampere}Strom}

\acro{j}[$j$]{\acrounit{-}Komplexe Zahl}
\acro{vec_j}[$\vec j$]{\acrounit{\ampere\per\square\meter}Elektrische Stromdichte}
\acro{vec_j_L}[$\vec j_L$]{\acrounit{\ampere\per\square\meter}Leistungsstromdichte}

\acro{k}[$k$]{\acrounit{-}Komplexe Wellenzahl}
\acro{k_12}[$k_{1\ldots2}$]{\acrounit{-}Komplexe Wellenzahl im Material 1 / 2}
\acro{vec_k}[$\vec k$]{\acrounit{-}Wellenzahlvektor}
\acro{vec_k_d}[$\vec k_d$]{\acrounit{-}Wellenzahlvektor der Teilwelle hinter Materialgrenzfläche}
\acro{vec_k_e}[$\vec k_e$]{\acrounit{-}Einfallender Wellenzahlvektor}

\acro{vec_k_r}[$\vec k_r$]{\acrounit{-}Reflektierter Wellenzahlvektor}

\acro{l}[$l$]{\acrounit{1}Modenzahl}
\acro{L}[$L$]{\acrounit{\meter}Länge}

\acro{m}[$m$]{\acrounit{1}Modenzahl}
\acro{M}[$M$]{\acrounit{-}Maßzahl der Variantenbewertung}

\acro{n}[$n$]{\acrounit{1}Modenzahl}
\acro{n_A}[$\vec n_A$]{\acrounit{1}Flächennormalenvektor}
\acro{N_w}[$N_w$]{\acrounit{1}Spulenwindungszahl}

\acro{p}[$p$]{\acrounit{\Dezibel}Leistungspegel}
\acro{P_12}[$P_{1\ldots2}$]{\acrounit{\watt}Leistung an Stelle 1 / 2}

\acro{q}[$q$]{\acrounit{\ampere\second}Elektrische Ladung}
\acro{Q}[$Q$]{\acrounit{-}Schirmfaktor}
\acro{Q_e}[$Q_e$]{\acrounit{-}Elektrischer Schirmfaktor}
\acro{Q_m}[$Q_m$]{\acrounit{-}Magnetischer Schirmfaktor}

\acro{r}[$r$]{\acrounit{-}Ortskoordinate im Kugelkoordinatensystem}
\acro{vec_r}[$\vec r$]{\acrounit{1}Ortsvektor}
\acro{r^F_n}[$r^F_n$]{\acrounit{\meter}Radius der n-ten Fresnelzone}
\acro{R_II}[$R_{\uproman{2}}$]{\acrounit{1}Reflektionsfaktor Betrachtungsfall \uproman{2}}
\acro{R_w}[$R_w$]{\acrounit{\Dezibel}Reflektionsverlust ebener Wellen}

\acro{S_w}[$S_w$]{\acrounit{\Dezibel}Schirmungseffektivität ebener Wellen}
\acro{S_HL}[$S_{HL}$]{\acrounit{\Dezibel}Hohlleiterdämpfung}

\acro{t}[$t$]{\acrounit{\second}Zeit}
\acro{t_S}[$t_S$]{\acrounit{\meter}Schirmdicke}

\acro{u(xt)}[$u(x,t)$]{\acrounit{-}Allgemeine Funktion}

\acro{v}[$v$]{\acrounit{\meter\per\second}Wellenausbreitungsgeschwindigkeit}

\acro{w}[$w$]{\acrounit{-}Gewicht der Variantenbewertung}

\acro{x}[$x$]{\acrounit{-}Ortskoordinate im karthesischen Koordinatensystem}
\acro{vec_x}[$\vec x$]{\acrounit{-}Ortskoordinatenvektor im kartesischen Koordinatensystem}
\acro{X}[$X$]{\acrounit{\Dezibel}Allgemeiner Feldgrößenpegel}
\acro{X_12}[$X_{1\ldots2}$]{\acrounit{-}Allgemeine Feldgröße an Stelle 1 / 2}

\acro{y}[$y$]{\acrounit{-}Ortskoordinate im karthesischen Koordinatensystem}

\acro{z}[$z$]{\acrounit{-}Ortskoordinate im karthesischen Koordinatensystem}
\acro{Z}[$Z$]{\acrounit{-}Wellenwiderstand}
\acro{Z_12}[$Z_{1\ldots2}$]{\acrounit{-}Wellenwiderstand im Material 1 / 2}

\end{acronym}
\newpage



\textit{Griechische Symbole} \\[.5\linespace]
\noindent\rule{\textwidth}{0.5pt}
\textbf{Symbol} \hspace{12.5mm} \textbf{Einheit} \hspace{10.5mm} \textbf{Bezeichnung} \\[-\linespace]
\noindent\rule{\textwidth}{0.5pt}

\begin{acronym}[Platzhalterwort]
\acro{alpha}[$\alpha$]{\acrounit{-}Dämpfungskonstante}

\acro{beta}[$\beta$]{\acrounit{-}Phasenkonstante}

\acro{gamma}[$\gamma$]{\acrounit{\radian}Winkel zwischen Hohlleiterwand und Wellenausbreitung}

\acro{delta}[$\delta$]{\acrounit{\meter}Eindringtiefe}
%\acro{Delta}[$\Delta$]{\acrounit{-}Laplace Operator}

\acro{varepsilon}[$\varepsilon$]{\acrounit{\ampere\second\per\volt\per\meter}Permittivität}
\acro{varepsilon_1}[$\varepsilon_{1\ldots2}$]{\acrounit{1}Permittivität im Material 1 / 2}
\acro{varepsilon_r}[$\varepsilon_r$]{\acrounit{1}Relative Permittivität}

\acro{theta}[$\theta$]{\acrounit{\radian}Winkelkoordinate im Kugelkoordinatensystem}

\acro{lambda}[$\lambda$]{\acrounit{\meter}Wellenlänge}
\acro{lambda_0}[$\lambda_0$]{\acrounit{\meter}Freiraumwellenlänge}
\acro{lambda_H}[$\lambda_H$]{\acrounit{\meter}Hohlleiterwellenlänge}
\acro{lambda_k}[$\lambda_k$]{\acrounit{\meter}Kritische Wellenlänge}

\acro{mu}[$\mu$]{\acrounit{\volt\second\per\ampere\per\meter}Magnetische Permeabilität}
\acro{mu_1}[$\mu_{1\ldots2}$]{\acrounit{\volt\second\per\ampere\per\meter}Magnetische Permeabilität im Material 1 / 2}
\acro{mu_r}[$\mu_r$]{\acrounit{1}Relative Permeabilität}

\acro{sigma}[$\sigma$]{\acrounit{\per\ohm\per\meter}Elektrische Leitfähigkeit}
\acro{sigma_r}[$\sigma_r$]{\acrounit{1}Relative Leitfähigkeit bezogen auf Kupfer}

\acro{phi}[$\phi$]{\acrounit{\radian}Winkelkoordinate im Kugelkoordinatensystem}
\acro{varphi_d}[$\varphi_d$]{\acrounit{\radian}Winkel einer Teilwelle nach Grenzübergang zur Flächennormalen}
\acro{varphi_e}[$\varphi_e$]{\acrounit{\radian}Einfallswinkel}
\acro{varphi_r}[$\varphi_r$]{\acrounit{\radian}Reflektionswinkel}

\acro{omega}[$\omega$]{\acrounit{\per\second}Kreisfrequenz}

\end{acronym}


\newpage



\subsection*{Konstanten}
\noindent\rule{\textwidth}{0.5pt}
\textbf{Symbol} \hspace{12.5mm} \textbf{Wert (gerundet)} \hspace{21mm} \textbf{Bezeichnung} \\[-\linespace]
\noindent\rule{\textwidth}{0.5pt}


\renewcommand{\acrounit}[2]{\acroextra{\makebox[55mm][l]{\SI{#1}{#2}}}} 

\begin{acronym}[Platzhalterwort]

\acro{Lichtgeschwindigkeit}[$c_0$]{\acrounit{2,99792e8}{\meter\per\second}Lichtgeschwindigkeit im Vakuum}

\acro{Eulerzahl}[$e$]{\acrounit{2,71828}{}Euler'sche Zahl}

\acro{Feldwellenwiderstand}[$Z_0$]{\acrounit{376,73}{\ohm}Feldwellenwiderstand des freien Raumes}

\acro{varepsilon}[$\varepsilon_0$]{\acrounit{8,85418e-12}{\ampere\second\per\volt\per\meter}Permittivität des freien Raumes}
\acro{mu_0}[$\mu_0$]{\acrounit{1,25664e-6}{\volt\second\per\ampere\per\meter}Magnetische Feldkonstante}
\acro{Pi}[$\pi$]{\acrounit{3,14159}{}Kreiszahl}


\end{acronym}


