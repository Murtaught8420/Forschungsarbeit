

\chapter{Einleitung}\label{cha:1}

Der Begriff der \acp{EMV} ist nicht nur beim Entwurf und Aufbau von Satelliten und anderen Raumfahrzeugen, sondern auch im terestrischen Bereich ein Designtreiber, der immer mehr an Bedeutung gewinnt. Aufgrund des Vordringens elektronischer Komponenten und Baugruppen in immer mehr Bereiche alltäglicher Anwendungen erhöht sich in gleichem Maße nicht nur die Anzahl der potentiell störanfälligen Geräte, sondern auch die Zahl elektromagnetischer Störquellen. \par
\vspace{\linespace}
Bei näherer Betrachtung wird schnell die Herausforderung der \ac{EMV} in ihrer Definition nach \cite{VDE_0870} deutlich: Während einerseits Geräte und Baugruppen vor Beeinflussung durch äußere Strahlungsquellen geschützt werden müssen und ihrerseits auch die Umgebung nicht unzulässig beeinflussen dürfen, müssen in steigendem Maße gleichzeitig auch Möglichkeiten drahtloser Kommunikation gewährleistet werden können. Um diese Anforderungen heute und in Zukunft mit der erforderlichen Güte zu erfüllen, sind nicht nur Maßnahmen zur Reduktion der Emission und Erhöhung der Störfestigkeit elektronischer Geräte nötig, sondern auch neue Materialien zur Beeinflussung des Koppelpfades elektrischer Systeme.
\par
\vspace{\linespace}
Am Institut für Luft- und Raumfahrttechnik der \ac{TUD} werden dafür \ac{EMV} Schutzfolien aus Kohlenstoffnanoröhrchen (CNT) \acused{CNT} entwickelt, mit deren Hilfe sich durch spezielle Verarbeitung ihrer Mikrostruktur die Dämpfungseigenschaften gegenüber elektromagnetischer Strahlung selektiv beeinflussen lassen. \par
\vspace{\linespace}
Um diese Eigenschaft im Rahmen der Entwicklung nachzuweisen und charakterisieren zu können, ist eine Vermessung der Schirmdämpfung innerhalb des ausgewählten Frequenzbereiches und der Feldeigenschaften unabdingbar. Dafür wird ein entsprechender Teststand benötigt, in dem die zur Erzeugung und Vermessung des Wellenfeldes verwendeten Hornantennen zusammen mit der Werkstoffprobe positioniert werden können und der eine möglichst störfeldfreie Messumgebung im Fernfeld der Antennen aufrechterhält.
\par
\vspace{\linespace}
Nach erfolgter Literaturstudie und Einarbeitung in den Themenkomplex der EMV-Messung und Hochfrequenztechnik soll dafür im Rahmen dieser Arbeit eine Messkabine zur Bestimmung der Dämpfungseigenschaften als Materialkenngröße von Folien und Schäumen im Fernfeld entworfen und aufgebaut werden. Unter Berücksichtigung des Frequenzbereiches von 0,8 bis \SI{18}{\giga\hertz} sollen dafür geeignete Bau- und Dichtungselemente ausgewählt werden, sodass die Anforderungen an die Störfestigkeit der Messumgebung, die Homogenität des Fernfeldes und die Wirtschaftlichkeit des Prüfstandes, eingehalten werden. Die bereits vorhandenen Hornantennen und die Materialproben sind weiterhin geeignet zu positionieren.
\par
\vspace{\linespace}
Ergebnis der Arbeit sollen nach der Durchführung entsprechender Messungen Aussagen über die Dämpfungseigenschaften von mindestens zwei Probetypen sein. Weiterhin ist neben der Dokumentation die Erstellung eines Manuals Ziel dieser Arbeit. Zusammen mit Messungen der Schirmdämpfung im Nahfeld einer Antenne, ergibt sich damit eine gesamtheitliche Aussage über das Dämpfungsverhalten der vermessenen Proben.