









%kein Unterschied feststellbar bei zwei Messungen ohne Probe an unterschiedlichen Tagen, da etwas gleiche Bedingungen und keine Veränderungen an Signalkabeln, etc zwischen Messungen nach Kalibration


%Ohne Erdung verstärkt sich Rauschen leicht, vor allem im hohen Frequenzbereich; kaum Unterschieschied, ob nur mit Schirm verbunden oder auch geerdet

%Absorber vor Antennen hat vor allem Auswirkungen im unteren Frequenzbereich
%gleiches gilt für Absorber unter Reflektor zur Reduktion von sekundären Wellenfronten --> Aussage aus \cite bestätigt, kein Einfluss ab ..GHz feststellbar

%Abkleben der Schnittkante des Reflektors hat geringen Einfluss, vor allem im hohen Frequenzbereich

%Peak aufgrund von Cut-Off und Beugungserscheinungen --> je besser Rest der Kurve aussieht, desto steiler wird Peak --> letztendlich bei 1,12965 GHz = 26,5385 cm Wellenlänge



%--> Schirmdämpfungsplots mit verschiedenen Anpassungen in ein Diagramm
        %-erste Messung, keine Anpassung
        %-Erdung
        %-normale Schrauben
        %-Absorber an Reflektor (hinten)
        %-Absorber an Reflektor (beidseitig)
        %-normale Schrauben und Abstand bei Messung ohne Probe
        %-keine Schrauben
        %-Absorber an PH



%An allen Varianten kann wie vermutet festgestellt werden, dass Reziprozität gilt, d.h. Werte für S12 und S21 sind bis auf das statistische Rauschen gleicht




% \begin{figure}[ht]
%     \centering
%     \includegraphics[page = 1, width = 0.99\textwidth, trim = 0cm 14.3cm 0cm 0cm, clip]{Abbildungen/Kapitel4/Messergebnisse/10k0x10k0-1.pdf}
%     \caption{Caption}
%     \label{fig:my_label3}
% \end{figure}